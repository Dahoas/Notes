\documentclass[10pt]{article}


\usepackage{amssymb,amsthm,amsmath}
\usepackage{enumerate}
\usepackage{graphicx,color}
\usepackage[hidelinks]{hyperref}
%\usepackage{refcheck}

\newcommand{\dd}{\mathrm{d}}
\newcommand{\E}{\mathbb{E}}
\newcommand{\1}{\textbf{1}}
\newcommand{\R}{\mathbb{R}}
\newcommand{\C}{\mathbb{C}} 
\newcommand{\Z}{\mathbb{Z}}
\newcommand{\N}{\mathbb{N}}
\newcommand{\p}[1]{\mathbb{P}\left( #1 \right)}
\newcommand{\scal}[2]{\left\langle #1, #2 \right\rangle}
\newcommand{\red}{\color{red}}

\DeclareMathOperator{\Var}{Var}
\DeclareMathOperator{\sgn}{sgn}

\usepackage[paper=a4paper, left=1.3in, right=1.3in, top=1in, bottom=1in]{geometry}
\linespread{1.3}
\pagestyle{plain}

\newtheorem{theorem}{Theorem}
\newtheorem{lemma}[theorem]{Lemma}
\newtheorem{corollary}[theorem]{Corollary}

\theoremstyle{remark}
\newtheorem{remark}[theorem]{Remark}


\newtheorem{conjecture}{Conjecture}

\theoremstyle{definition}
\newtheorem{defn}[theorem]{Definition}
\newtheorem{exmp}[theorem]{Example}


\title{\vspace{-3em}Some Results}



\begin{document}

Consider the monotone strictly increasing sequence of nonnegative numbers $X = (x_i)_{i=1}^L$, $L \in \N$. We consdier the exponential sum defined by $S_X(z) = \sum_{n=1}^L e^{x_nz}$ and consider its zeroes. 


\section{On Concavity of Geometric Sequence}

This argument needs some small fixing(I misplace some terms in the computation) but the core idea should still work: show concavity of $F(t) = log(t\sum e^(k t))$ by looking at level sets of second derivative of $log(\sum e^(kt)$ and bounding decay via integral approximation. Lots of annoying algebra. But clearly exponentially decaying.

\section{On Imaginary Zeroes of Arithmetic Progressions and Symmetries}

\begin{theorem}
	Let the sequence X above be an arbitrary arithmetic progression, ie. of the form $x_1 = d$, $x_2 = d+c$,..., $x_L = d + (L-1)c$ for arbitrary $d \in \R, c > 0$. Then $S_X(z)$ has zeroes only on the imaginary axis.
\end{theorem}

\begin{proof}
 Write 
	\begin{align*}
		S_X(z) = \sum_{n=1}^L e^(x_n z) = 0 \iff e^{x_1z}(\sum_{n=1}^L e^{(x_n-1x_1)z}) = 0 \iff \sum_{n=1}^L e^{(x_n-x_1)z} = 0
	\end{align*}

	since $e^{x_1z}$ has no zeroes. So wlog we may assume $x_1 = 0$, since the translation still results in an arithmetic sequence. Then we sum

	\begin{align*}
		\sum_{n=1}^L e^{x_1 z} &= \sum_{n=1}^L e^{(n-1)cz} = \sum_{n=0}^{L-1}(e^{cz})^{n} = \frac{e^{Lcz}-1}{e^{cz}-1}
	\end{align*}

	So 

	\begin{align*}
		S_X(z) = 0 \implies \frac{e^{Lcz}-1}{e^{cz}-1} = 0 \implies e^{Lcz} = 1 \implies z = ib
	\end{align*}

	for some $b \in \R$. In fact we must have $Lcz = 2\pi n \implies z = \frac{2 \pi n}{L c}$ for some $n \in \Z$
\end{proof} 

Using Rouche's Theorem we can extend this result to the symmetric case:

\begin{theorem}
	Consider the sequence $Y = -X \cup X + c$ where X is arithmetic as defined above and $c \in \R$ is an arbitrary translation. Then $S_Y(z)$ has zeroes only on the imaginary axis.
\end{theorem}

\begin{proof}
	Via shifting it suffices to consider $c = 0$. Then we argue wlog for $z \in \C$ s.t. $Re(z) > 0$, $|S_X(z)| \geq |S_{-X}(z)|$. Since we already know $S_X$ has no zeroes in an arbitrarily small ball around $z \in \C, Re(z) > 0$, this completes the proof via Rouche. 

	Simply compute

	\begin{align*}
		& S_X(z) = \frac{e^{Lcz}-1}{e^{cz}-1}, S_{-X}(z) = \frac{e^{-Lcz}-1}{e^{-cz}-1} \\
		& |S_{-X}| \leq |S_{X}| \iff |\frac{e^{-Lcz}-1}{e^{-cz}-1}| \leq |\frac{e^{Lcz}-1}{e^{cz}-1}| \iff |\frac{e^{-Lcz}}{e^{-cz}}||\frac{e^{Lcz}-1}{e^{cz}-1}| \leq |\frac{e^{Lcz}-1}{e^{cz}-1}| \\
		& \iff 1 \leq |\frac{e^{Lcz}}{e^{cz}}| = e^{Re(z)(L-1)c}
	\end{align*}
\end{proof}

\section{On Imaginary Zeroes of Complement Sequences}

Let the sequence X defined above be integer valued. We prove 

\begin{theorem}
	$S_X(z) = 0 \implies Re(z) = 0 \iff \forall n \in [L], x_i \in X \iff x_L - x_n \in X$ or X is a translation of such a sequence.
\end{theorem}

\begin{proof}

	Write 
	\begin{align*}
		S_X(z) = \sum_{n=1}^L e^(x_n z) = 0 \iff e^{x_1z}(\sum_{n=1}^L e^{(x_n-1x_1)z}) = 0 \iff \sum_{n=1}^L e^{(x_n-x_1)z} = 0
	\end{align*}

	since $e^{x_1z}$ has no zeroes. So wlog we may assume $x_1 = 0$. Then make the change of variables $z = log(z')$ where we note complex log is surjective 
	since $z = log(z') \iff e^z = z'$. Also $z = ib \iff log(z') = ib \iff z' = e^(ib) \implies z' \in \mathbb{S}^1$. Compute

	\begin{align*}
		S_X(log(z')) = 1 + \sum_{n=2}^L e^{log((z')^{x_n})} = 1 + \sum_{n=2}^L (z')^{x_n} = P_X(z')
	\end{align*}

	which is a polynomial in terms of z'. Thus $S_X(z)$ has an imaginary root $\iff$ $P_X(z')$ has a root on the complex unit circle. So $S_X$ has strictly imaginary axis roots iff $P_X$ has strictly unit modulus roots. It is known that the only polynomials which have strictly unit modulus roots are the palindromic polynomials($a_i = a_{n-i}$) and the anti-palindromic polynomials($a_i = -a_{n-i}$)\cite{MR}. It s clear our polynomial P cannot be anti-palindromic, so to have all roots on $\mathbb{S}^1$ it must be palindromic. 

	The rest does not work since palindromic polynomials need not have all unit modulus roots.

\end{proof}

\begin{remark}
	This result includes a farily wide class of sequences, including integer arithmetic progressions and their symmetries. Call this class the set of complement sequences $\mathcal{C}$. So in particular this implies the random variable defined as $P(X = k) = 1/2L$ for $k \in \{-L,...,-1\} \cup \{1,...,L\}$ is type L(and hence Ultra-Sub Gaussian). This also fully characterizes discrete integer valued uniform type L random variables.
\end{remark}

\section{On The Zeroes of Other Types of Sequences}


Don't yet have a full characterization but we know through testing

\begin{itemize}
	\item The geometric sequence $(q^i)_{i=1}^L$ does not have strictly imaginary zeroes
	\item Binomial coefficient sequences(the first half) do not have strictly imaginary zeroes
\end{itemize}

so this breaks for the full log-concave class.

\section{Other Ideas}

I came across a theorem from Control Theory? called the Jury Test\cite{KB} which tells us when a polynomial has zeroes outside of the unit disc by looking at operations on the coefficients. I don't think this is useful anymore since we now have a complete characterization via these palindromic polynomials, but I'm not sure. It's proved using Rouche's theorem so it may be useful.

I think we should be able to extend this proof invovling palindromic polynomials to a wider class via some shifting arguments, but I haven't thought about how very thoroughly yet.

My notation changed several times over the course of writing this out, sorry about that.

%\nocite{*}


\begin{thebibliography}{9}


\bibitem{MR} Markovsky. I, Shodhan. R,
Palindromic Polynomials, Time-Reversible Systems, and Conserved Quantities. https://eprints.soton.ac.uk/266592/1/Med08.pdf

\bibitem{KB} Keel. L, Bhattarcharyya. S,
A New Proof of the Jury Test. https://ieeexplore.ieee.org/document/703305
%\bibitem{Z-2} 
%https://mathoverflow.net/questions/208349

\end{thebibliography}

\end{document}



