\documentclass[10pt]{article}


\usepackage{amssymb,amsthm,amsmath}
\usepackage{enumerate}
\usepackage{graphicx,color}
\usepackage[hidelinks]{hyperref}
%\usepackage{refcheck}

\newcommand{\dd}{\mathrm{d}}
\newcommand{\E}{\mathbb{E}}
\newcommand{\1}{\textbf{1}}
\newcommand{\R}{\mathbb{R}}
\newcommand{\C}{\mathbb{C}} 
\newcommand{\Z}{\mathbb{Z}}
\newcommand{\N}{\mathbb{N}}
\newcommand{\scal}[2]{\left\langle #1, #2 \right\rangle}
\newcommand{\red}{\color{red}}
\newcommand{\shift}{\vdash}
\newcommand{\pP}{\mathcal{P}}
\newcommand{\lL}{\mathcal{L}}

\DeclareMathOperator{\Var}{Var}
\DeclareMathOperator{\sgn}{sgn}

\usepackage[paper=a4paper, left=1.3in, right=1.3in, top=1in, bottom=1in]{geometry}
\linespread{1.3}
\pagestyle{plain}

\newtheorem{theorem}{Theorem}
\newtheorem{lemma}[theorem]{Lemma}
\newtheorem{corollary}[theorem]{Corollary}

\theoremstyle{remark}
\newtheorem{remark}[theorem]{Remark}


\newtheorem{conjecture}{Conjecture}

\theoremstyle{definition}
\newtheorem{definition}[theorem]{Definition}

\theoremstyle{prop}
\newtheorem{prop}[theorem]{Prop}

\theoremstyle{Corollary}
\newtheorem{Corollary}[theorem]{Corollary}


\title{\vspace{-3em}Log}



\begin{document}


\section{Polya Schur Notes}

\subsection{Multiplier Sequences}

\begin{definition}
	$T = \{\gamma_k\}_{k=0}^{\infty} \subseteq \R$ is a multiplier sequence if, when arbitrary polynomial $p(z) = \sum_{k=0}^n a_k z^k$ ahs only real zeroes, $T[p(z)] = \sum_{k=0}^n \gamma_k a_kz^k$ has only real zeroes. This extends to arbitrary $f \in \lL \pP$ via $T[f(z)] = \sum_{k=0}^{\infty} \gamma_k a_k z^k$
\end{definition}

Power series analog of universal factors. Whereas universal factors correspond to integral representations(as we will see below).

Classification:

\begin{theorem}
	T is a multiplier sequence $\iff$

	1. The function
	\[
		\phi(z) = T[e^z] = \sum \gamma_k \frac{z^k}{k!} \in \lL \pP^+
	\]

	2. Jensen Polynomials 
	\[
		T[(1+z)^n] = \sum_{k=0}^n {n \choose k} \gamma_k z^k \in \lL \pP^+
	\]
\end{theorem}

Also we have the connection via Laguerre

\begin{theorem}
	If $\phi(z) \in \lL\pP$ and $\phi$ has no zeroes in $(0,n)$ then $\phi$ evaluated at the integers $[n]$ acts as a multiplier sequence on polynomials of degree up to n. 
\end{theorem}


\subsection{Universal Factors}


\begin{definition}\textit{Universal Factors}
	
	Let $K(t) \in O(e^{-|t|^b})$, $b > 2$. Then $\phi(t)$ is a universal factor of class r if for $K$ with $\int_{-r}^r K(t) e^{izt}dt \in \lL \pP$ then
	\begin{align*}
		\int_{-r}^r \phi(t)K(t)e^{izt}dt \in \lL \pP
	\end{align*}
	is entire with only real zeroes
\end{definition}


\begin{theorem}
	Let $\phi(z) = \sum_{k=0}^{\infty} \gamma_k z^k$ with $\phi, f \in \mathcal{L}\mathcal{P}$. Then the differential operator $\phi(D)$ when acting on f has real zeroes.
\end{theorem}

So differentiation via functions in the $\lL \pP$ of functions in $\lL \pP$ is closed.

\begin{prop}

Let $0 < r \leq \infty$. If $\phi(iz) \in \lL \pP$ then $\phi(t)$ is a universal factor of class r. 
\end{prop}

Does this go the other way? Probably not. Is there a restriction allowing us to go other way?

\begin{theorem}
	If $K(t) = e^{-\alpha_1 t^2} K_0(t)$, $\alpha_1 > 0$ and $K_0(t)$ is s.t.
	\[
		K_0(z) = ce^{\alpha_0 z^2 + bz} z^m \prod_{k=1}^{\infty} (1+z/z_k) e^{-z/z_k}
	\]
	for $b,c,iz_k \in \R$ with $\alpha_1 > \alpha_0$ then $f(z) = \int_{-\infty}^{\infty}K(t) e^{izt}dt \in \lL \pP$
\end{theorem}	

We think of $K_0$ as $\phi_X$ for some $x \in \lL$. Then if we regard this as a density for some rv Y, $y \in \lL$. What is this operation called?

\begin{theorem}
	If real analytic $\phi(t)$ is universal factor, then $\phi(iz) \in \lL \pP$. 
\end{theorem}

So universal factors, ie. functions that preserve real zeroes via pointwise product with the kernel, yield type $\lL$ characteristics(assuming positivity). Analytically universal factors are exactly those functions in $\lL \pP$.

\begin{definition}\textit{Mellin Transform}
	
	If $K(t) : [0,\infty) \to \R$ integrable on $[0,\infty)$ then 
	\[
		H(z) = \int_0^{\infty}K(t)t^{z-1}dt
	\]
	is the Mellin transform
\end{definition}

Connects log of a random varible with with the random variable(scaling the density by an exponential). Sometimes easier to show property of Mellin transform instead of fourier transform to show some $f \in \lL \pP$ .

The proofs of the above characterizations heavily lie on differential operators. This is how we can view the pointwise product. 


\subsection{Polya-Schur Theory}

Reading of \cite{BB}.

Key questions in polya-schur theory:

Let $U \subseteq \C$ with $Z(U)$ the set of all complex polynomials whose zeroes lie in $U$. Set $Z_n(U)$ to be the subset of such polynomials of degree at most n. 

\begin{itemize}
	\item What is the set of all linear transformations $T : Z(U) \to Z(U) \cup \{0\}$
	\item What is the set of all linear transformations $T : Z_n(U) \to Z(U) \cup \{0\}$
\end{itemize}

So what linear operators preserve the zeroes of such polynomials? \cite{BB} solves this problem for $U$ a line, circle, closed half plane, closed disk, complement of open disk. 

Note in particular this addresses problems of what linear operators preserve the zeroes of functions $\lL \pP$, via approximation by Jensen polynomials: Apply operator to Jensen polynomials and converge zeroes via hurwitz.

\begin{theorem}(Polya-Schur Theorem)

	Let $\lambda : \N \to \R$ be a sequence of real numbers and $T: \R[z] \to \R[z]$ be the corresponding diagonal linear operator given by $T(z^n) = \lambda(n) z^n$. Define $\Phi(z)$ as

\[
	\Phi(z) = \sum_{k=0}^{\infty} \frac{\lambda(k)}{k!}z^k
\]
Then the following are equivalent:
	
	i) $\lambda$ is a multiplier sequence

	ii) $\Phi$ defines an entire function which is the limit, uniformly on compact sets, of polynomials with only real zeros of the same sign

	iii) Either $\Phi(z)$ or $\Phi(-z)$ is entire of the form $C z^n e^{az}\prod_{k=1}^{\infty}(1+\alpha_k z)$ with $a,\alpha_k \geq 0$ and $\sum \alpha_k < \infty$

	iv) For all non-negative integers n th polynomials $T[(z+1)^n]$ is hyperolic with all zeroes same sign
\end{theorem}


This is the classical (partial) characterization of Polya-Schur. Note the multiplier sequences correspond to diagonal matrices on the monomial basis. The problem was solved more generally in the following form:

Set $\mathcal{H}_1(\C)$ to be the set of stable polynomials in n varibles. 

\begin{theorem}
	Let $n \in \N$ and $T : \R_n[z] \to \R[z]$ be linear operator. Then T preserves hyperbolicity $\iff$ either

	1) T has range of dimension at most two and is of the form $T(f) = \alpha(f) P + \beta(f) Q$ for $\alpha,\beta \in \R_n[z] \to \R$ are linear functionals and $P,Q \in \mathcal{H}_1(\R)$ have interlacing zeroes

	2) $T[(z+w)^n] \in \mathcal{H}_2(\R)$ 

	3) $T[(z-w)^n] \in \mathcal{H}_2(\R)$
\end{theorem}

\begin{thebibliography}{9}

\bibitem{PH} Hallum P. Zeroes of Entire Funtions Represented by Fourier Transforms. 

\bibitem{BB} Borcea J, Branden P. Polya-schur Master Theorems for Circular Domains and Their Boundaries


\bibitem{MR} Markovsky. I, Shodhan. R,
Palindromic Polynomials, Time-Reversible Systems, and Conserved Quantities. https://eprints.soton.ac.uk/266592/1/Med08.pdf

\bibitem{KB} Keel. L, Bhattarcharyya. S,
A New Proof of the Jury Test. https://ieeexplore.ieee.org/document/703305
%\bibitem{Z-2} 
%https://mathoverflow.net/questions/208349



\end{thebibliography}

\end{document}



