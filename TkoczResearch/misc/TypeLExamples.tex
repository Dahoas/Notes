\documentclass[10pt]{article}


\usepackage{amssymb,amsthm,amsmath}
\usepackage{enumerate}
\usepackage{graphicx,color}
\usepackage[hidelinks]{hyperref}
%\usepackage{refcheck}

\newcommand{\dd}{\mathrm{d}}
\newcommand{\E}{\mathbb{E}}
\newcommand{\1}{\textbf{1}}
\newcommand{\R}{\mathbb{R}}
\newcommand{\C}{\mathbb{C}} 
\newcommand{\Z}{\mathbb{Z}}
\newcommand{\N}{\mathbb{N}}
\newcommand{\p}[1]{\mathbb{P}\left( #1 \right)}
\newcommand{\scal}[2]{\left\langle #1, #2 \right\rangle}
\newcommand{\red}{\color{red}}
\newcommand{\shift}{\vdash}
\newcommand{\lL}{\mathcal{L}}

\DeclareMathOperator{\Var}{Var}
\DeclareMathOperator{\sgn}{sgn}

\usepackage[paper=a4paper, left=1.3in, right=1.3in, top=1in, bottom=1in]{geometry}
\linespread{1.3}
\pagestyle{plain}

\newtheorem{theorem}{Theorem}
\newtheorem{lemma}[theorem]{Lemma}
\newtheorem{corollary}[theorem]{Corollary}

\theoremstyle{remark}
\newtheorem{remark}[theorem]{Remark}


\newtheorem{conjecture}{Conjecture}

\theoremstyle{definition}
\newtheorem{defn}[theorem]{Definition}
\newtheorem{exmp}[theorem]{Example}
\newtheorem{prop}[theorem]{Prop}


\title{\vspace{-3em}A Study of Khintchine Type Inequalities for Random Variables}
\author{Alex Havrilla}



\begin{document}

\maketitle

\section{Known Examples of Type L Random Variables}

Relatively few examples of Type L random variables are known. The majority we do have follow from results of Polya in his study of kernels producing strictly real zeroes of fourier transforms of the form:

\begin{align*}
	\phi(z) = \int_{\R} K(x)cos(zx)dx
\end{align*}

for some kernel $K : \R \to \R$. This can naturally be interpreted as the inverse fourier transform of a random variable $X$ with density $K$. And (assuming symmetry and nice gaussanity conditions) if $\phi$ has strictly real zeroes then we know $\phi(iz) = \E e^{-zX}$ has strictly imaginary zeroes and hence X is type L.

\subsection{Polya's Examples}

All of these examples can be found in Polya's \textit{Problems in Analysis} but the experience of retrieving them(and their proofs) is somewhat time consuming. Hopefully this presentation is somewhat less so. 

\begin{theorem}[Decreasing Concave Density(173)] \label{CCDNSTY}
	Let $X$ be a symmetric continuous random variable distributed on $[0,1]$ density $f$ s.t. $f',f'' < 0$. Then $X \in \lL$.
\end{theorem}

\begin{theorem}[L1 Bounded Derivative(175)] \label{LBDER}
	Let $X$ be a symmetric continuous random variable distributed on $[0,1]$ with density f s.t. $|f(1)| \geq \int_0^1 |f'(t)|dt$. Then $X \in \lL$. Note in particular this works for the case $f$ is increasing.
\end{theorem}

\begin{theorem}[Exponential Density(170)] \label{EXPDEN}
	Let $\alpha$ be even integer greater than 2. Then if $X$ a symmetric continuous random variable with density of the form $e^{-t^{\alpha}}$ then $X \in \lL$
\end{theorem}

\begin{theorem}[Exponential Product Density(161)] \label{EXPPDEN}
	Let $1 > \alpha \geq 0, 0 < \alpha_1 \leq \alpha_2 \leq ...$ and reciprocal convergent. Then if $g(z)= e^{-\alpha z} (1-\frac{z}{\alpha_1})(1-\frac{z}{\alpha_2})...$ we have for symmetric X with density $e^{-t^2} g(-t^2)$ then $X \in \lL$.
\end{theorem}

\begin{theorem}[Bessel Function(159)] \label{BF}
	The symmetric continuous random variable X with density $\frac{2}{\pi\sqrt{1-t^2}}$ in $\lL$.
\end{theorem} 

\begin{theorem}[Large nth Coefficient(27)] \label{LNC}
	Suppose X a discrete integer valued symmetric distribution. If $p_0 + 2p_1 +... + 2p_{n-1} < 2p_n$ then $X \in \lL$. 
\end{theorem}

%Am I missing one of the examples Tkocz went through? With increasing density or whatever?

\subsection{Newman's Examples}

Newman, who initiated our study in Type L random variables, produced some examples as well. 

\begin{theorem}[Arithmetic Sequences] \label{them:AS}
	Let the sequence X above be an arbitrary arithmetic progression, ie. of the form $x_1 = d$, $x_2 = d+c$,..., $x_L = d + (L-1)c$ for arbitrary $d \in \R, c > 0$. Then $S_X(z)$ has zeroes only on the imaginary axis.
\end{theorem}

\begin{theorem}[Uniform(Newman 7)] \label{UNI}
	Let $X$ be random variable with density $\frac{d\mu}{dy} = 1$ if $|y| \leq A$ and 0 otherwise. $A > 0$. Then $X \in \lL$.
\end{theorem}

\begin{theorem}[Newman (8)] \label{N8}
	Density $(1-y^2)^{(d-2)/2}$ with $|y| \leq 1$ and 0 otherwise. For $d > 0$. 
\end{theorem}

\begin{theorem}[Newman (9)] \label{N9}
	Density $e^{-\lambda cosh(y)}$, $\lambda >0$
\end{theorem}

\begin{theorem}[Newman (10)] \label{N10}
	$e^{-ay^4-by^2}$ with $a>0$
\end{theorem}

\subsection{Other Examples}

\begin{theorem}[Enestrom-Kakeya] \label{them:EK}
	If $X$ integer valued symmetric with $0\leq p_0 \leq 2 p_1 \leq ... \leq 2p_n$ with $p_n >0$ then $X \in \lL$.
\end{theorem}

\begin{theorem}[Absolute Value] \label{them:ABS}
	Let $a_0,a_1,...,a_n \in \R$ with $|a_0| + .. | a_{n-1}| \leq |a_n|$ then the trig polys $p_c(z) = \sum_{k=0}^n a_k cos(kz)$ and the sin one have only real zeroes
\end{theorem}

\end{document}



