\documentclass[11pt]{article}
%you can look for fun LaTeX packages to use hereasdf

\usepackage{amsmath}
\usepackage{amssymb}
\usepackage{fancyhdr}
\usepackage{amsthm}

\usepackage{graphicx}
\usepackage{dcolumn}
\usepackage{bm}

%fun commands for fun sets
%make sure to use these in math mode
\newcommand{\Z}{\mathbb{Z}}
\newcommand{\R}{\mathbb{R}}
\newcommand{\N}{\mathbb{N}}
\newcommand{\C}{\mathbb{C}}
\newcommand{\m}{\mathcal{M}}
\newcommand{\Tt}{\mathcal{T}}
\newcommand{\pa}{\partial}
\newcommand{\dD}{\mathcal{D}}
\newcommand{\E}{\mathbb{E}}



\oddsidemargin0cm
\topmargin-2cm    
\textwidth16.5cm   
\textheight23.5cm  

\newcommand{\question}[2] {\vspace{.25in} \hrule\vspace{0.5em}
\noindent{\bf #1: #2} \vspace{0.5em}
\hrule \vspace{.10in}}
\renewcommand{\part}[1] {\vspace{.10in} {\bf (#1)}}

\newcommand{\myname}{Alex Havrilla}
\newcommand{\myandrew}{alumhavr}
\newcommand{\myhwnum}{Hw 1}

\newtheorem{prop}{Prop}
\newtheorem{lemma}{Lemma}
\newtheorem{theorem}{Theorem}
\theoremstyle{remark}
\newtheorem*{rem}{Remark}
\newtheorem*{defi}{Def}
\newtheorem*{apps}{Application}
\newtheorem*{quest}{Question}
\newtheorem*{ans}{Answer}
\newtheorem*{interest}{Interesting}
\newtheorem*{theme}{Theme}
\newtheorem*{back}{Background}
\newtheorem*{example}{Example}

\setlength{\parindent}{0pt}
\setlength{\parskip}{5pt plus 1pt}
 
\pagestyle{fancyplain}
\lhead{\fancyplain{}{\textbf{HW\myhwnum}}}      % Note the different brackets!
\rhead{\fancyplain{}{\myname\\ \myandrew}}
\chead{\fancyplain{}{\mycourse}}

\title{$1 \leq p \leq 2$ For Type L with enough Gausannity}

\linespread{1.3}

\begin{document}

\maketitle

\section{Results}

Let $X^{(b)}$ be a random variable with density given by $\frac{1}{2\pi}(1-b+bx^2)e^{-x^2/2}$. Set $X^{(b)}_p = X^{(b)}/||X^{(p)}||_p$. 

We rewrite with where S is a sum of iid $X^{(b_i)}_p$

\begin{align*}
	\E|X^{(b)}_p+S|^p = \frac{\E ||X^{(b)}|_p+S|^p+\E||X^{(b)}_p|-S|^p}{2}
\end{align*}

Then using technique of interlacing densities and the convexity of 
\begin{align*}
	h(x) = |x^{1/p}+1|^p + |x^{1/p}-1|^p
\end{align*}

via Lemma 12 from 

\begin{verbatim}
	https://www.math.cmu.edu/~ttkocz/mypapers/mathematics/khintchBpn.pdf
\end{verbatim}

we show for $b > 0$
\begin{align*}
	\E h(X^{(b)}_p) \geq \E h(X^{(0)}_p)
\end{align*}

which gives us a khintchine type lower bound since $X^{(0)}$ is gaussian. So it suffices to show the difference of densities $f^{(b)}_p(x) - f^{(0)}_p(x)$ has at most two zeroes. Note we already know there are at least 2 via both being probability distributions and agreeing on pth moments. We compute

\begin{align*}
	f^{(b)}_p(x) =\frac{1}{\sqrt{2}}e^{-1/2(\pi^{-1/2/p}\sqrt{2}&x((1+bp)\Gamma(1+p/2))^{1/p})^{2/p}}\pi^{-1/2-1/2p}(2^{p/2}(1+bp)\Gamma(1+p/2))^{1/p}\\
	&(1-b+b(\pi^{-1/2/p}\sqrt{2}x(1+bp)\Gamma(1+p/2)^{1/p})^{2/p})
\end{align*}

ie. setting $||X^{(b)}||_p = C^{(b)}_p = (\frac{\sqrt{2^p}}{\sqrt{\pi}}(1+bp)\Gamma(1+p/2))^{1/p}$ which is clearly increasing in b we have

\begin{align*}
	f^{(b)}_p(x) = C^{(b)}_p (1-b+b(C^{(b)}_p x)^2)e^{-(C^{(b)}_px)^2/2}
\end{align*}

and in particular

\begin{align*}
	f^{(0)}_p(x) = C_p^{(0)}e^{-(C_p^{(0)}x)^2/2}
\end{align*}

The taking logs we know $f^{(b)}_p = f^{(0)}_p$ when

\begin{align*}
	& log(C^{(b)}_p) +log(1-b+b(C^{(b)}_px)^2)-(C^{(b)}_px)^2/2=log(C^{(0)}_p) -(C^{(0)}_px)^2/2 \iff \\
	&  log(1-b+b(C_p^{(b)}x)^2) = (C_p^{(b)}^2-C_p^{(0)}^2)x^2/2+A_p^{(b)}
\end{align*}

where we clearly get two intersections when the affine term has a positive slope. Since the pth moment increasing in b, we have this. 

Note this also establishes comparisons between $1>b_1,b_2 >0$ since pth moment increasing in b.

\section{Continuations}

\begin{enumerate}
\item Comparing arbitrary $b_1,b_2$ instead $b,0$? - This seems to be resolved
\item $p \in (0,1)$? 
\end{enumerate}

\end{document}

