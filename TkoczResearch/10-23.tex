\documentclass[10pt]{article}


\usepackage{amssymb,amsthm,amsmath}
\usepackage{enumerate}
\usepackage{graphicx,color}
\usepackage[hidelinks]{hyperref}
%\usepackage{refcheck}

\newcommand{\dd}{\mathrm{d}}
\newcommand{\E}{\mathbb{E}}
\newcommand{\1}{\textbf{1}}
\newcommand{\R}{\mathbb{R}}
\newcommand{\C}{\mathbb{C}} 
\newcommand{\Z}{\mathbb{Z}}
\newcommand{\N}{\mathbb{N}}
\newcommand{\scal}[2]{\left\langle #1, #2 \right\rangle}
\newcommand{\red}{\color{red}}
\newcommand{\shift}{\vdash}
\newcommand{\pP}{\mathcal{P}}
\newcommand{\lL}{\mathcal{L}}

\DeclareMathOperator{\Var}{Var}
\DeclareMathOperator{\sgn}{sgn}

\usepackage[paper=a4paper, left=1.3in, right=1.3in, top=1in, bottom=1in]{geometry}
\linespread{1.3}
\pagestyle{plain}

\newtheorem{theorem}{Theorem}
\newtheorem{lemma}[theorem]{Lemma}
\newtheorem{corollary}[theorem]{Corollary}

\theoremstyle{remark}
\newtheorem{remark}[theorem]{Remark}


\newtheorem{conjecture}{Conjecture}

\theoremstyle{definition}
\newtheorem{definition}[theorem]{Definition}

\theoremstyle{prop}
\newtheorem{prop}[theorem]{Prop}

\theoremstyle{Corollary}
\newtheorem{Corollary}[theorem]{Corollary}


\title{\vspace{-3em}Log}



\begin{document}

\section{10/23}

Notes

\section{10/29}

\textbf{Questions:}

How do we extend to khintchine inequality?

\section{11/19}

\textbf{Goals:}

Show distribution inequality. Involving lambert w functions.

\textbf{Plan:}

Break into cases: 

Note first over $t \in (t_{lm},1]$ we know $G(t) \geq F(t)$ since $t_{lm} = 1/2e^{-3/2}$ is a local optimium for G and above this threshold is it strictly dominated. (For some reason graph of $p=2$ not reflective of this? Should be dominated regardless of distribution of mass). So dif $D = F - G$ negative.

Then note from $(t_+,t_{lm})$ we seek to show the derivative D' negative.

From $(0,t_+)$ seek to show D postive.

This will show we have exactly one crossing from + to -.

Issue: $t_+$ seems to depend on p.

\textbf{Confusions:}

The plots don't seem to be reflecting a switch from positive to negative given a fixed variance? But it must be there?

Upon closer inspection just seems to go negative very quickly for large p. So G very quickly dominates F as p increases. Increasing p concentrates mass at 0? Makes sense cause we know gaussian F dominates G near 0(so G larger since measuring under mass).

Also all plots regardless of p should be negative at $t_{lm}$ but this doesn't seem to be the case. Makes me worry the formulation is incorrect.

Why does it suffice to compute measure for $x > 0$ in distribution argument?

\subsection{The Lambert Function}

$W(x)e^{W(x)} = x$. But multiple solutions to $W(xe^x)$ (if negative).

Implcitly defined. But similar to logs and exponentials:

\begin{align*}
	\lim_{x \to \infty} \frac{W_0(x)}{ln(x)} = 1\\
	\lim_{x \to 0^-} \frac{W_{-1}(x)}{ln(-x)} = 1
\end{align*}



\begin{thebibliography}{9}

\bibitem{HH} Hasani. Inequalities for Lambert Function. \begin{verbatim}https://www.emis.de/journals/JIPAM/images/107_07_JIPAM/107_07_www.pdf \end{verbatim}

\bibitem{PH} Hallum P. Zeroes of Entire Funtions Represented by Fourier Transforms. 

\bibitem{BB} Borcea J, Branden P. Polya-schur Master Theorems for Circular Domains and Their Boundaries


\bibitem{MR} Markovsky. I, Shodhan. R,
Palindromic Polynomials, Time-Reversible Systems, and Conserved Quantities. https://eprints.soton.ac.uk/266592/1/Med08.pdf

\bibitem{KB} Keel. L, Bhattarcharyya. S,
A New Proof of the Jury Test. https://ieeexplore.ieee.org/document/703305
%\bibitem{Z-2} 
%https://mathoverflow.net/questions/208349



\end{thebibliography}

\end{document}



