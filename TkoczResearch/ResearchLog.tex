\documentclass[11pt]{article}
%you can look for fun LaTeX packages to use hereasdf

\usepackage{amsmath}
\usepackage{amssymb}
\usepackage{fancyhdr}
\usepackage{amsthm}

\usepackage{graphicx}
\usepackage{dcolumn}
\usepackage{bm}

%fun commands for fun sets
%make sure to use these in math mode
\newcommand{\Z}{\mathbb{Z}}
\newcommand{\R}{\mathbb{R}}
\newcommand{\N}{\mathbb{N}}
\newcommand{\C}{\mathbb{C}}
\newcommand{\m}{\mathcal{M}}
\newcommand{\Tt}{\mathcal{T}}
\newcommand{\pa}{\partial}
\newcommand{\dD}{\mathcal{D}}
\newcommand{\E}{\mathbb{E}}



\oddsidemargin0cm
\topmargin-2cm    
\textwidth16.5cm   
\textheight23.5cm  

\newcommand{\question}[2] {\vspace{.25in} \hrule\vspace{0.5em}
\noindent{\bf #1: #2} \vspace{0.5em}
\hrule \vspace{.10in}}
\renewcommand{\part}[1] {\vspace{.10in} {\bf (#1)}}

\newcommand{\myname}{Alex Havrilla}
\newcommand{\myandrew}{alumhavr}
\newcommand{\myhwnum}{Hw 1}

\newtheorem{prop}{Prop}
\newtheorem{lemma}{Lemma}
\newtheorem{theorem}{Theorem}
\theoremstyle{remark}
\newtheorem*{rem}{Remark}
\newtheorem*{defi}{Def}
\newtheorem*{apps}{Application}
\newtheorem*{quest}{Question}
\newtheorem*{ans}{Answer}
\newtheorem*{interest}{Interesting}
\newtheorem*{theme}{Theme}
\newtheorem*{back}{Background}
\newtheorem*{example}{Example}

\setlength{\parindent}{0pt}
\setlength{\parskip}{5pt plus 1pt}
 
\pagestyle{fancyplain}
\lhead{\fancyplain{}{\textbf{HW\myhwnum}}}      % Note the different brackets!
\rhead{\fancyplain{}{\myname\\ \myandrew}}
\chead{\fancyplain{}{\mycourse}}

\title{Research Log}

\linespread{1.3}

\begin{document}



\maketitle

\section{12/26}

\subsection{Inequalities}

\section{12/29}

\subsection{Lambert Function Inequalities}

\begin{verbatim}
	/home/alex/Desktop/Notes/Winter 2020/LatexForms/lambertinequalities.pdf
\end{verbatim}

\section{1/7}

I continue trying to find good inequalities for lambert.

\subsection{Lambert}

\begin{defi}
	Pochamer symbol is rising factorial
\end{defi}

\subsection{Tkoczs Algebraic Approach: type-L-bimod}

\begin{rem}
	Suprising $\E|\sum \sqrt{a_j} X_j|^p - \E |X_1|^p = C_p' P(q)$ is a polynomial
\end{rem}

\begin{quest}
	Does the supremal gaussian case work out to a polynomial?
\end{quest}

\begin{ans}
	Yes this is the last part of the writeup.
\end{ans}

\begin{quest}
	How do nonnegative coefficnets reduce to p=2 case?
\end{quest}

\begin{ans}
	test
\end{ans}

\begin{quest}
	What is $q^{(0)}$?
\end{quest}
\begin{ans}
	1 and $q^{(1)} = q$
\end{ans}

\section{1/13}

\subsection{Easy Regime}

\begin{remark}
	If we require sum of coefficients $c=1$ we can always write type L RV as sum of quadratic components(assuming it has appropriate gaussian weight).
\end{remark}

\begin{quest}
	For quadratic case validity as a RV for some characteristic is determined by $0 < b < 1$ on $(1-bt^2)$. Are there conditions for higher even orders powers? Example of type L rv which takes this form but cannot be written as sum of quadratics?
\end{quest}

\begin{quest}
	Why does tkocz say we require coefficients to sum to 1 for type L?
\end{quest}

\section{1/24}

\begin{quest}
	Constraints on type L density? Must calculate IFT of characteristic. Type L poly gauss density?
\end{quest}

\begin{quest}
	Want to extend to results for characteristic $(1-bx^2)e^{-x^2/2}$ for $0 < b < 1$ to type L random variables in some particular class. 

	Do sums of form of two $b_1,b_2$ satisfy a khintchine inequality
\end{quest}

\begin{quest}
	Need to precisely understand scaling properties
\end{quest}

\begin{remark}
	Note densities of form are essentially rescalings ie. changes in variance of $(1-x^2)e^-x^2/2$
\end{remark}

\begin{quest}
	How to prove default khintchine inequality $p \geq 3$?
\end{quest}

\begin{quest}
	Can we define an alternative class since FT of gaussian times hermite is still gaussian times hermite? Is the span of all nonnegative hermite polynomials all nonnegative polynomials?sss
\end{quest}

\begin{quest}
	Can inequalities be extended to larger class via some kind of density argument? Is class dense in some larger class?

	Also still lots of questions from discrete case to address. Could lead to results in continuosu case via density?
\end{quest}

\begin{remark}
	Gaussian-polynomial densities seem closely related to hermite polynomials and solutions to schrodinger wave equation
\end{remark}

\section{Dump from OneNote}

\subsection{Examples of Type L}

\begin{example}
	Examples:*In general really must be working with reciprocal sequences since roots must be all on unit circle(even in positive symmetrized case) since constant is 1 whichi s product of roots

	Arithmetic sequences and sums
	
	Arbitrary Symmetrizations: 
		2.18 from thesis
	
	(PIA 173): Symmetric rv with concave density
	
	Arithmetic with strictly increasing probailities
	
	Arbitrary sequences of length 4(Seq4)
	
	Also some kind of result for strictly inc. measure from problems in analysis



Nonexamples:
	
	Binomial Coefficients
	
	Geometric Random variables
	
	Complement Sequence: (0,1,3,7,9,10)
	
	Not every sequence is symmetrizable: (1,7,9,11)

\end{example}

\subsection{Sequence Testing}

Self-Reciprocal real roots always come in pairs of four(if not real)
	If real pairs of two
	
Could also just look for integral which has roots on unit circle

We must be looking at palindromic polynomials(even in positive case) Since:
	Constant coefficient is 1 which is product of norms and inverses. So everything must be on unit circle, which implies palindromic.
	
Symmetric(up to shifting) iff palindromic
	Clearly palindromic implies symmetric. But goes other way(just shift palindrome to center )
		Odd sequences have mass at 0?
		
Moral: Mass should not be concentrated at middle, but instead at endpoints

TRYING TO FIND POLYS NOT PRODUCTS OF ARITHMETIC POLYS:

	1 + x^2 + x^3 + x^5 + x^15 + x^17 + x^18 + x^20
	(x + 1)^2 (x^2 + 1) (x^2 - x + 1)^2 (x^4 - x^3 + x^2 - x + 1) (x^8 + x^7 - x^5 - x^4 - x^3 + x + 1)
	Roots all on unit circle
	
	Adding:
	
		Breakers:
		But adding x,x^19 breaks it(most roots still on unit circle but some 3 reciprocals come off)
		x^4,x^(16) breaks it
		6,14 breaks it
		7,13
		8,12
		9,11
		10: Jacking up coefficient on this really fucks with roots(completely off circle, very reciprocated)
		
		Preservers:
			4,16+7,13 works(this turns into +/- product of arithmetic sequences)
		
		
	Removing:
		
		Breakers:
			2,18
			3,17
			5,15
		
		Preservers:
			Removing any pair results in a grouping of 4(which we proved in many ways works)
			
	Adding in 2 groups:
		
		Breaks:
			8,12+9,11
			4,16+9,11
			4,16+6,14
			4,16+8,12
			4,16+10
		
		Preserves:
			4,16+7,13(arithmetic product)
			4,16+1,19(not arithmetic product)
			
	
	
	Maybe need to add terms in groups of 4?(like roots?)
	Should also try modulating coefficient of middle term(if even power, odd seq length)
	Should try to describe class of arithmetic sequences are convolution(what does this generate?)
		Corresponds to multiplication of arithmetic polynomials
	What do minuses represent?(On polynomials?)
	When does the middle term (n/2) not break things?
	
	Some polynomials seem to arise as products of -1/0/1 polynomials
		Maybe this class should be studied
		
Dividing Out Roots of Unity:

\subsection{Real Polys with all roots on unit circle}

https://pdfs.semanticscholar.org/b2d7/eac15216b322bca452ae07660d8f24a87a0a.pdf
Palindrome Polyonomials with Roots on the Unit Circle

	Every even degree 0/1 polynomials has at least one root on unit circle
	
Note On Derivatives of 0/1 Polynomials:
	Suppose P is derivative of Q where P is 0/1 polynomial. If all the roots of Q are on the unit circle, then the roots of Q must also be on unit circle.
	
Maybe consider process of dividing out unimodular roots until none are left?
	Need to show each subsequent polynomial has another unimodular root.
	?How to divide 0/1 polynomial by unimodular root so result is still 0/1 polynomial?
	How can we multiply a 0/1 polynomial by unimodular roots to get a 0/1 polynomial
	
Instead of finding the roots, start with the unimodular roots and see what classes of 0/1 polynomials these create
	Products of polynomials with unimodular roots have unidmodular roots
	
	A polynomial is not 0/1 if it has no unimodular roots
	
	?When is a product of 0/1 polynomials 0/1?
		For all k there exists at most one way of summing to it(in sequence, one from each)
		
		So if sequences are far apart in gap size
			If g(X)= smallest gap
			Need g(X)>y_N
			
		Also this obviously holds for arithmetic polys
	Are certain products of roots of unity 0/1 polys?
			
This is all motivated by fact that every 0/1 poly has unit rootd

I might also try to take a more probabilistic perspective(less analysis oriented)

\subsection{Zeros in Unit Disk}

file:///C:/Users/Alex/Downloads/Acta_math-2009.pdf
Polynomials with All zeros on unit circle

A polynomial has unit roots iff 
	i) Self inversive
	ii) P' has all roots on or in circle
Basically Cohn's Theorem
	Note this is proved with Rouche's theorem

So just need to show P' has roots in unit circle

Tools for All roots in Unit Circle:
	Jury test and variants
	Enestrom Kakeya
	https://math.stackexchange.com/questions/867408/to-prove-this-complex-polynomial-has-all-zeros-on-unit-circle

Common Method of Proof: Chebyshev Transform

Sufficient condtion (5)  |A_m+B |≥∑_(k=1)^(m−1)▒|B+A_k−A_m |   tells us we can have any two terms(B=1)

Polynomials with 0/1 Coefficients: https://www.math.tamu.edu/~terdelyi/papers-online/CA.pdf
	A lot of work with bounding number of zeroes in unit circle or in some region
		Polygons in Unit circle
		Strips
		
Root Counting Stories:
	Polya investigates multiplicity of 0/1 polys roots at 1
	Improved by schur
	Use Jensen type stuff
	
Cohn's Theorem: https://en.wikipedia.org/wiki/Cohn%27s_theorem
Nth degree reciprocal poly has as many roots in open unit disk as reciprocal polynomial of its derivative

\subsection{July 28,31}

Written notes on polynomials with symmetric zeroes

\subsection{August 3}

Notes on Zeroes, mahler measure, looking for generators of type L via addition of random variables. Cyclotomic polynomials.

\subsection{August 6}

Looking at subset sum problems. 

Look into Conway-Guy sequence
Sum Packing Problem and Conway Guy: ams.org/journals/proc/1996-124-12/S0002-9939-96-03653-2/S0002-9939-96-03653-2.pdf
https://oeis.org/A005318
a construction for sets of integers with distinct subset sums
file:///C:/Users/Alex/Downloads/1341-PDF%20file-1420-1-10-20120117.pdf
	Suggests methods for generating subset sum distinct sequences using Conway Guy Sequence algo
	
Any n-sequence generate a Type-L sequence: need reverse characterization. When is a sequence(prob distribution) writable as subset sum:
	Seq. needs to be length 2^n, or can be extended. 
	?Always possible to extend sequence to write as subset sum?
	What does the subset sum transformation do?
	?What kinds of properties does subset sum transformation preserve?
		Very nonlocal operation
		
	
?Approximate inequalities as a result of "almost type L" sequences?
	Generated by approximation algorithms?
	Probabilistic inequality: holds with some probability
	
	
Probabilistically subset summing is adding together n bernoulli ranom variables with 1/2 masses. 
	Class doesn't seem that interesting: Just leads to variations of binomial distributions
	Subset summing really not that interesting since it all follows from type L closed under addition
	?To extract useful characterization need characterization of subset sum decomposale sequences)?
	

Polya's Problems in Analysis:

\section{August 17 and After}

Looking at fourier zeroes

\section{1/25}

\subsection{Questions for Tkocz}

\begin{quest}
	Representation of gaussian as infinite product? All I can find is 
	\begin{verbatim}
		https://www.researchgate.net/publication/327057578_The_Exponential_Function_and_its_Infinite_Product
	\end{verbatim}

	Perhaps something to be done using the gamma function(which has an infinite product) and stirling approximation? Or sin or something.
\end{quest}

\begin{quest}
	Can we always assume roots sum to 1? Why?
\end{quest}

\subsection{Finding Higher Order Atoms}

\begin{remark}
	If any higher order atoms do exist they must have sum of roots higher than variance of gaussian. since otherwise we can factor their characterstic in a way which is a sum of two lower orders. So with normalization larger than 1.
\end{remark}

\begin{remark}
	Nonexamples(which are not random variables):
	\begin{itemize}
		\item IFT of $(1-x^2)(1-4x^2)e^{-x^2/2}$ yields $e^{-y^2/2}(8-19y^2 +4y^4)$ which is not nonnegative.
	\end{itemize}

	Nonexamples(which are type L but sums of lower orders):
	\begin{itemize}

	\end{itemize}

	Examples(characteristic poly factors symmetrically):
	\begin{itemize}
		\item IFT of $e^{-x^2/2}(3-6x^2+x^4) = e^{-x^2/2}(y^2-\sqrt{6}-3)(y^2+\sqrt{6}-3)$ is $e^{-y^2/2}y^4$
		\item IFT of $e^{-x^2/2}(-15 + 45 x^2 - 15 x^4 + x^6)$ is $e^{-y^2/2}y^6$ 
	\end{itemize}

	Seems to be difficult to find higher order atoms.
\end{remark}

\begin{remark}
	More generally for densities of the form $e^{-x^2/2}x^{2n}$ we have the Fourier transorm as function of some hypergeometric thing.
	\begin{align*}
		2^{n+1}\Gamma[1/2+n]HypergeometricF1[1/2+n,1/2,-y^2/2]
	\end{align*} 
\end{remark}

\begin{interest}
	Could be densities of form $e^{-x^2/2}x^{2n}$ for backbone form atoms? Extremal cases?
\end{interest}

\begin{remark}
	Fourier transform of even real valued function is even real valued.
\end{remark}

\begin{quest}
	Is every even gaussian poly density type L? Earlier I thought type L densities closed under FT. Is this at least true for this class?

	Note if the density is not even then it cannot possibly be type L. 
\end{quest}

\begin{ans}
	no not every even gaussian poly density is type L because we know the form of the characteristic explictly(poly times gaussian and even).  how do we know roots are real? 

	But of course the IFT of arbitrary even characteristic is not necessarily density.

	However class not closed under FT cause characteristics not necessarily nonnegative. More accurate to say gaussian times poly form closed under FT.
\end{ans}

\begin{quest}
	How do the roots grow for $x^{2n}e^{-x^2/2}$?
\end{quest}

\section{2/4}
 
None

\section{2/5}

None

\section{2/6}

Set $I_n(y)  = \mathcal{F}_+(x^n e^{-x^2/2}) = \int_{\R} x^ne^{-x^2/2}e^{-ixy}dx$.From integration by parts we can obtain $I_n(y) = (n-1)I_{n-2}(y) - iyI_{n-1}$

\section{2/7}

\textbf{TAG:} Research

\subsection{Definitions}

\begin{defi}
	$Hypergeometric1F1[a,b,z]=\sum_{k=0}^{\infty} \frac{(a)_k}{(b)_k}\frac{z^k}{k!}$
\end{defi}

\subsection{Investigating Extremal powers larger than 2}

Set $I_n(y)  = \mathcal{F}_+(x^n e^{-x^2/2}) = \int_{\R} x^ne^{-x^2/2}e^{-ixy}dx$.From integration by parts we can obtain $I_n(y) = (n-1)I_{n-2}(y) - iyI_{n-1}$

\begin{remark}
	On fourier transforms of this type:
	\begin{verbatim}
		https://math.stackexchange.com/questions/108633/fourier-transform-of-gaussian-times-polynomial-to-a-high-power
	\end{verbatim}
\end{remark}

We know

\begin{align*}
	I_{n+1}(y) = i I_n'(y)
\end{align*}

\begin{quest}
	How does interpolation of higher order densities work? Can we get khintchine inequalities for them?
\end{quest}

\end{document}

