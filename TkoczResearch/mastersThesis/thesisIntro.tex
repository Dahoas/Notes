\documentclass[10pt]{article}


\usepackage{amssymb,amsthm,amsmath}
\usepackage{enumerate}
\usepackage{graphicx,color}
\usepackage[hidelinks]{hyperref}
%\usepackage{refcheck}

\newcommand{\dd}{\mathrm{d}}
\newcommand{\Pp}{\mathbb{P}}
\newcommand{\E}{\mathbb{E}}
\newcommand{\1}{\textbf{1}}
\newcommand{\R}{\mathbb{R}}
\newcommand{\C}{\mathbb{C}} 
\newcommand{\Z}{\mathbb{Z}}
\newcommand{\N}{\mathbb{N}}
\newcommand{\p}[1]{\mathbb{P}\left( #1 \right)}
\newcommand{\scal}[2]{\left\langle #1, #2 \right\rangle}
\newcommand{\red}{\color{red}}
\newcommand{\shift}{\vdash}
\newcommand{\lL}{\mathcal{L}}
\newcommand{\norm}[1]{||#1||}

\DeclareMathOperator{\Var}{Var}
\DeclareMathOperator{\sgn}{sgn}

\usepackage[paper=a4paper, left=1.3in, right=1.3in, top=1in, bottom=1in]{geometry}
\linespread{1.3}
\pagestyle{plain}

\newtheorem{theorem}{Theorem}
\newtheorem{lemma}[theorem]{Lemma}
\newtheorem{corollary}[theorem]{Corollary}

\theoremstyle{remark}
\newtheorem{remark}[theorem]{Remark}


\newtheorem{conjecture}{Conjecture}

\theoremstyle{definition}
\newtheorem{defn}[theorem]{Definition}
\newtheorem{exmp}[theorem]{Example}


\title{\vspace{-3em}A Survey of Khintchine Type Inequalities for Random Variables}
\author{Alex Havrilla}



\begin{document}

\maketitle

\begin{abstract}
	We present a collection of Khintchine type inequalities for random variables and some applications in convex geometry and functional analysis. In particular we present new results for symmetric discrete distributions and for the class of Type L random variables.
\end{abstract}

\newpage

\section{Introduction}

The study of Khintchine inequalities examines inequalities for the moments of sums of independent identically distributed(iid) random variables. Recall the pth moment of a random variable $X$ as

\begin{align*}
	||X||_p = (\E|X|^p)^{1/p}
\end{align*}

where $\E$ denotes expectation.

The classical Khintchine inequality compares for $0 < p,q < \infty$ the pth and qth moments of sums of iid random signs with coefficients $a_i \in \R$
\begin{align*}
	S = \sum_i^n a_i \epsilon_i
\end{align*}

where we recall $\Pp(\epsilon = 1) = \Pp(\epsilon = -1) = \frac{1}{2}$. 

\begin{theorem}Khintchine's Inequalities:

	$\forall p,q > 0 \exists C_{p,q} > 0$ and $\forall n, a \in \R^n$ we have
	\begin{align*}
		||S||_p \leq C_{p,q} ||S||_q
	\end{align*}
\end{theorem}


Of particular interest is finding sharp constants $C_{p,q}$. This is easy when $p < q$, as then $C_{p,q} = 1$. However the case $p > q$ is difficult. In general we only know comparisons for arbitrary p and $q=2$ and for $p,q$. 

There are a number of ways to generalize moment comparison inequalities of this form. Optimal constants for distributions other than the random sign have been studied, yielding the same constants as in the random sign case. Generalizations to vector coefficients have also been made with applications in Banach space theory.

\section{Khintchine Inequalities for Various Distributions}

This section seeks to present a collection of Khintchine type results with optimal constants for variously distributed random variables. We start with progenitor case of random signs.

\subsection{Random Signs}

In this section we deal strictly with sums of random signs $\epsilon_i$. First we establish constants $C_{p,2}$ and $C_{2,p}$ do exist for $||S||_p \leq C_{p,2} ||S||_2$, $ ||S||_2 \leq C_{2,p}||S||_p$ where $S = \sum a_i \epsilon_i$. We argue as in \cite{LT}. This is a natural starting point since second moment of sums is easily computable:

\begin{align*}
	\norm{S}_2 = \sqrt{\E S_a^2} = \sqrt{\E\sum_i a_i^2 \epsilon_i + \E\sum_{i,j}a_ia_j \epsilon_i\epsilon_j} = \sqrt{\sum_i a_i^2}
\end{align*}

ie. the second moment of the sum is just the sum of squares of the coefficients. Often without loss of generality we will choose $a_i$ s.t. $\sum_i a_i^2 = 1$.  

We adopt the convention $A_p = C_{2,p}$ and $B_p = C_{p,2}$. First we establish the existence of such constants. 

\begin{theorem}Khinchin's Inequality:
	
	For $0 < p < \infty, \exists A_p,B_p > 0$ s.t. for $a \in \R^n$,
	\begin{align*}
		A_p\sqrt{\sum_i a_i^2} \leq \norm{S_a}_p \leq B_p \sqrt{\sum_i a_i^2}
	\end{align*}
\end{theorem}

We argue as in \cite{LT}.

\begin{proof}
	Via homogeneity we may suppose $\sum_i a_i^2 = 1$. Then via layer cake and chebyshev we may write:
	\begin{align*}
		\E |\sum_i \epsilon_i a_i|^p = \int_0^{\infty} \Pp(|\sum_i \epsilon_i a_i | > t) dt^p \leq 2 \int_0^{\infty} e^{-t^2/2}dt^p = B_p^p
	\end{align*}
	For the other side write
	\begin{align*}
		1 = \E(\sum_i \epsilon_i a_i)^2 \leq (\E |\sum_i \epsilon_i a_i|^p)^{2/3}(\E |\sum_i \epsilon_i a_i|^{6-2p})^{1/3} \leq (\E|\sum_i \epsilon_i a_i|^p)^{2/3} B_{6-2p}^{2-2p/3}
	\end{align*}
	at which point we are done.
\end{proof


So we know such comparisons are possible but we seek to find optimal constants. The case for $p \geq 3$ turns out to be relatively easy via some convexity arguments by Young in \cite{Y}. 

\begin{theorem}Young:

For $3 \leq p < \infty$ we have $B_p = 2^{1/2}(\Gamma((p+1)/2)/\sqrt{\pi})^{1/p}$	
\end{theorem}

The proof crucially relies on the convexity of a certain function h for $p \geq 3$ which inspires relevant ideas later. 

\begin{lemma}
	Define for $z \in \C$
	\begin{align*}
		h_p(z,t) = |1-tz|^{p-2} + |1+tz|^{p-2} 
	\end{align*}
	For $p \geq 3$ $h_p$ is convex in t.
\end{lemma}

%Supply a proof of this later!!!

In \cite{H} Haagerup concluded the search for random signs by treating $0 < p < 2$ and $2 < p < 3$ and which establishes the following constants:


\begin{align*}
	A_p = 
	\begin{cases}
		2^{1/2-1/p} & 0 < p \leq p_0 \\
		2^{1/2}(\Gamma((p+1)/2)/\sqrt{\pi})^{1/p} & p_0 < p < 2\\
		1 & 2 \leq p < \infty
	\end{cases}
\end{align*}

and 

\begin{align*}
	B_p = 
	\begin{cases}
		1 & 0 < p \leq 2\\
		2^{1/2}(\Gamma((p+1)/2)/\sqrt{\pi})^{1/p} & 2 < p < \infty
	\end{cases}
\end{align*}

where $p_0$ is the solution to $\Gamma((p+1)/2) = \sqrt{\pi}/2$ in $[1,2]$. Szarek \cite{S} showed $A_1 = 1/\sqrt{2}$. Haagerup extended this result for $0 < p < p_0$ and and established optimality of $B_p$ for $2 < p <3$. Note this is seen to be sharp by way of Central Limit Theorem, as the sum $\sum_i \frac{1}{\sqrt{n}}\epsilon_i \to g \sim N(0,1)$ which achieves the $B_p$ in the limit. Note the $1/\sqrt{2}$ is achieved by the sum of two random signs.

While it would be nice to give complete proofs of these sharp constants, Haagerups arguments in particular are quite technical and not entirely enlightening. Instead we elect to present a proof via Nazarov, F. Podkorytov \cite{NP} which relies on crucial elements of Haagerups proof while using distribution functions to "simplify" some more technical details. This treats the $0 < p <2$ case. In \cite{M} Mordhorst treats $2 < p < 3$ using the same technique.

Now we must present a lemma on distribution functions.

\begin{lemma}
	Let $Y > 0$, $f,g : \mathcal{M} \to [0,Y]$ be any two measurable functions on $(\mathcal{M},\mu)$. Let $F_*$ and $G_*$ be their modified distribution functions. Assume both $F_*(y)$ and $G_*(y)$ are finite for every $y \in (0,Y)$. Assume also $\exists ! y_0$ s.t. $F_*-G_* = 0$. Furthermore at $y_0$ we need a change in sign from $+$ to $-$. Let $S = \{s > 0: f^s - g^s \in L^1(\mathcal{M},\mu)\}$. Then
	\begin{align*}
		\phi(s) = \frac{1}{sy_0^s}\int_{\mathcal{M}}f^s - g^s d\mu
	\end{align*}
	is monotone increasing on S. 
\end{lemma}

%Should I include proof???

We will use this to show an integral inequality in the following proof.

Here we present a sketch of Nazarov and Podkorytov's approach:

\begin{proof}
	First we reduce the case $0 < p < p_0$ to $p = p_0$. Let $S = \sum_i a_i \epsilon_i$ and suppose $\E|S|^p \geq 2^{p_0 - 2/2}$ where $\sum_ a_i^2 = 1$. Via holder we have
	\begin{align*}
		\E|S|^{p_0} \leq (\E|S|^p)^{2-p_0/2-p}(\E|S|^2)^{p_0-p/2-p} = (\E|S|^p)^{2-p_0/2-p}
	\end{align*} 
	since $\E|S|^2 = 1$ by assumption. So we only consider now $p_0 \leq p \leq 2$. 

	As in Haagerup's argument, we crucially rely on an integral representation of these moments
	\begin{align*}
		\E|S|^p = C_p\int_0^{\infty}\frac{1-\prod_{k=1}^n cos(a_k u)}{u^{p+1}}
	\end{align*}

	where we note the product of cosines the product of characteristic functions of the scaled random signs.

	Using this representation we can reduce to question to an integral inequality. We have the comparison via AM-GM
	\begin{align*}
		\prod_{k=1}^n cos(a_k u) \leq \prod_{k=1}^n |cos(a_k u)| \leq \sum_{k=1}^n a_k^2|cos(a_k u)|^{q/a_k^2} = 1 - \sum_{k=1}^na_k^2(1- |cos(a_k u)|^{1/a_k^2})
	\end{align*}

	Define 
	\begin{align*}
		I_p(s) = C_p\int_0^{\infty}(1-|cos(\frac{u}{\sqrt{s}})|^2)\frac{du}{u^{p+1}}
	\end{align*}

	Then we can write
	\begin{align*}
		\E|S|^p \geq \sum_{k=1}^na_k^2 I_p(\frac{1}{a_k^2})
	\end{align*}
	so it suffices to treat $I_p$. We know for $I_p(2) = 2^{p-2/2}$ via the integral representation. For $lim_{s \to \infty}I_p(s)$ we may apply DCT to see get $C_p \int_0^{\infty}(1-e^{-u^2/2})\frac{du}{u^{p+1}}$ as well $\lim_{s \to \infty}I_p(s) = 2^{p-2/2}\frac{\Gamma(p+1/2)}{\Gamma(3/2)}$ via a CLT arugment. Then showing $I_p(s) \geq I_p(\infty)$ would allow us to conclude. Ie write

	\begin{align*}
		H(p,s) = \int_0^{\infty}(e^{-sx^2/2} - |cos(x)|^s)\frac{dx}{x^{p+1}} \geq 0 
	\end{align*}

	It is this kind of integral inequality to which we can apply the distribution function lemma. We must compute
	\begin{align*}
		&F_*(y) = \mu\{x > 0 : e^{-x^2/2} < y\} = \frac{1}{p}2ln(1/y)^{-p/2}\\
		&G_*(y) = \mu\{x > 0: |cos(x)| < y\} = \frac{1}{p}\sum_{k=0}^{\infty}\frac{1}{(\pi k + arccos(y))^p} - \frac{1}{(\pi k + \pi - arccos(y))^p}
	\end{align*}

	and show the difference only changes sign once on the interval $(0,\pi/2)$. This is shown using an extension of the difference to the complex plane.

	After this it remains to treat the case of large coefficient, ie. one of the $a_k > \frac{1}{2}\sum_{k=1}^na_k^2$. To do this we go by induction. The observation
	\begin{align*}
		\E|S|^p(a) = \frac{\E|S|^p(a^+) + \E|S|^p(a^-)}{2}
	\end{align*}

	where $a^+ = (a_1,...,a_{n-2},a_{n-1}+a_n)$ and $a^-$ analagously allows us to absorb large coefficients into smaller term expressions. Thus letting $\phi_p(x)$ be a particular convex function we end up with the estimate
	\begin{align*}
		\E|S|^p(a) = \frac{\E|S|^p(a^+) + \E|S|^p(a^-)}{2} \geq A_p \frac{\phi_p(x^+)+\phi_p(x^-)}{2} \geq A_p\phi_p((x^++x^-)/2) = A_p \phi_p(x) \geq A_p \psi_p(x)
	\end{align*}

	where $\psi_p(x) = (1+x)^{p/2}$ and $\phi$ dominates $\psi$ and $x = a_2^2 + ... + a_n^2$. Which allows us to conclude.

	%Need to clean this end up.
\end{proof}

Moving forward it is good to keep these sharp constants in mind, and the techniques used to prove them. Via the CLT arguments we see in $B_p$ does not change if we change the distribution of the terms being summed. Similarly convexity arguments seem to continue to work well for $p \geq 3$. 

This concludes our discussion of Khintchine type inequalities for random signs. We now move on to discussing generalizations with other distributions.

\subsection{Ultra Sub-Gaussian Random Variables}

This section examines a result via Nayar and Oleszkiewicz in \cite{NO} which significantly generalizes the class of random variables considered and achieves comparisons for all even moments via log-concavity.

First we discuss the notions of Ultra Sub-Gaussanity and Strong Log Concavity of a Random Variable. Recall

\begin{defn}
	A sequence $(a_i)_{i=0}^{\infty}$ of non-negative real numbers is called \textit{log-concave} if $a_i^2 \geq a_{i-1}a_{i+1}$.
\end{defn}

Then we can define the notion of \textit{Ultra Sub-Gaussanity} for $\R^n$ valued random vectors from \cite{NO}.

\begin{defn}
	$\R^n$ valued $X$ is \textit{Ultra sub-Gaussian} if $X=0$ or X is rotation invariant, has finite moments, and has gaussian log-concave even moments ie. $a_i = \E \norm{X}^{2i}/\E \norm{G}^{2i}$ are log-concave.
\end{defn}

The results crucially use Walkup's Theorem concerning preservation of log-concavity under convolution:

\begin{theorem}
	Let $(a_i)_{i=0}^{\infty}$ and $(b_i)_{i=0}^{\infty}$ be two log-concave sequences of positive real numbers. We define:
	\begin{align*}
		c_n = \sum_{i=0}^n {n \choose i} a_i b_{n-i}
	\end{align*}

	Then the $(c_n)_{n=0}^{\infty}$ is log-concave
\end{theorem}

This tells us the collection $\textbf{ULCS}$ of ultra log-concave sequences is closed under convolution:

\begin{defn}
	$(a_i)_{i=0}^{\infty}$ is ultra log-concave if $(i! a_i)_{i=0}^{\infty}$ is log-concave.
\end{defn}

Given X Ultra Sub-Gaussian we can extract a khintchine type inequality of the following form.

\begin{theorem}
	Let n,d positive integers and $p > q \geq 2$ even integers. If $X_1,...,X_n$ independent $\R^n$ valued random vectors are ultra sub-Gaussian then
	\begin{align*}
		(\E |S|^p)^{1/p} \leq \frac{(\E|G|^p)^{1/p}}{(\E|G|^q)^{1/q}}\E (|S|^q)^{1/q}
	\end{align*}

	where $S = \sum_i X_i$
\end{theorem}

The proof rests on some lemmas we highlight now.

\begin{lemma}
	Let $\Prod: \R^n \to \R$ be projection to first coordinate. For $p > 0$ assume X rotation invariant random vector on $\R^n$ with finite pth moment. Then
	\begin{align*}
		\frac{\E|\Prod X|^p}{\E |G|^p} = \frac{\E||X||^p}{\E||G||^p}
	\end{align*}
\end{lemma}

Notably the class of ultra sub-gaussian random variables is closed under sums.

\begin{lemma}
	If $X,Y \in \textbf{USG}$ are independent random vectors then $X+Y \in \textbf{USG}$.
\end{lemma}

\begin{proof}
	We may assume X and Y nonzero constant. Setting
	\begin{align*}
		&a_i = \E||X||^{2i}/\E||G||^{2i} = \E(\prod X)^{2i}/\EG^{2i}\\
		&b_i = \E||Y||^{2i}/\E||G|^{2i} = \E(\prod Y)^{2i}/\E G^{2i}\\
		&b_i = \E||X+Y||^{2i}/\E||G|^{2i} = \E(\prod (X+Y))^{2i}/\E G^{2i}
	\end{align*}
	and then note
	\begin{align*}
		c_n  = \frac{1}{(2n-1)!!}\sum_{i=0}^{n}{2n \choose 2i} \E(\Prod X)^{2i} \E(\Prod Y)^{2n - 2i} = \sum_{i=0}^n \frac{(2n)!!}{(2i)!!(2n-2i)!!}a_ib_{n-i} = \sum_{i=0}^n {n \choose i}a_i b_{n-i}
	\end{align*}
	and we conclude with Walkup's theorem.
\end{proof}

We must also note distributions producting with nonnegative random variables to a standard gaussian must ultra sub-gaussian.

\begin{lemma}
	Assume an $\R^d$ valued random vector X and non-negative random variable R are independent and that $R \cdot X$ has distribution $N(0,I d_d)$. Then X is ultra sub-gaussian
\end{lemma}

Now we establish the Khintchine-type inequality for ultra sub-gaussian random variables.

\begin{proof}
	Note first the constant is optimal as indicated by the central limit theorem.

	If we show $S$ ultra log-concave then we know the sequence $a_k = \E||S||^{2k}/\E||G||^{2k}$ is log-concave which gives us our desired result since then we know $(a_s^{1/s})_{s=1}^{\infty}$ is non-increasing and in particular $a_{p/2}^{2/p} \leq a_{q/2}^{2/q}$. But S is clearly ultra log-concave as it is the sum of ultra log-concave random variables. So we are done.
\end{proof}

These results in turn relate to our discussion of type L random variables, where we discover every Type L random variable is ultra sub-gaussian.

\subsection{Type L Random Variables}



\subsection{Discrete Symmetric Distributions}

We now consider the generalization of a random sign to a symmetric discrete random variables uniformly distributed outside 0. Ie. consider the generalization to $X ~ \{-L,,..,0,...,L\}$ with some mass $\Pp(X = 0) = \rho_0$ and otherwise uniformly distributed on the $\{-L,...,-1\} \cup \{1,...,L\}$. We have the following results.

First note for small enough mass at 0 the random variables is ultra sub-gaussian and thus we have khintchine type-inequalities for all even moments.

\begin{theorem}\label{thm:USG}
Let $\rho_0 \in [0,1]$ and let $L$ be a positive integer. Let $X_1, X_2,\dots$ be i.i.d. copies of a random variable $X$ with $\p{X=0} = \rho_0$ and $\p{X = -j} = \p{X = j} = \frac{1-\rho_0}{2L}$, $j = 1,\dots,L$. Then $X$ is ultra sub-Gaussian if and only if $\rho_0 = 1$, or
\begin{equation}\label{eq:USG-rho}
\rho_0 \leq 1 - \frac{2}{5}\frac{3L^2+3L-1}{(L+1)(2L+1)}.
\end{equation}
If this holds, then, consequently, for positive even integers $q > p \geq 2$, every $n \geq 1$ and reals $a_1,\dots,a_n$, we have
\begin{equation}\label{eq:Khin-even}
\left(\E\left|\sum_{i=1}^n a_iX_i\right|^q\right)^{1/q} \leq C_{p,q}\left(\E\left|\sum_{i=1}^n a_iX_i\right|^p\right)^{1/p}
\end{equation}
with $C_{p,q} = \frac{[1\cdot 3\cdot\ldots \cdot (q-1)]^{1/q}}{[1\cdot 3\cdot\ldots \cdot (p-1)]^{1/p}}$ which is sharp.
\end{theorem}

 Here we first need to recall the classical notions of majorisation and Schur-convexity. Given two nonnegative sequences $(a_i)_{i=1}^n$ and $(b_i)_{i=1}^n$, we say that $(b_i)_{i=1}^n$ \emph{majorises} $(a_i)_{i=1}^n$, denoted $(a_i) \prec (b_i)$ if
\[
\sum_{i=1}^n a_i = \sum_{i=1}^n b_i \qquad \text{and} \qquad \sum_{i=1}^k a_i^* = \sum_{i=1}^k b_i^* \ \text{ for all } \ k = 1,\ldots,n,
\]
where $(a_i^*)_{i=1}^n$ and $(b_i^*)_{i=1}^n$ are nonincreasing permutations of $(a_i)_{i=1}^n$ and $(b_i)_{i=1}^n$ respectively. For example, $(\frac{1}{n},\frac{1}{n},\dots,\frac{1}{n}) \prec (a_1,a_2,\dots,a_n) \prec (1,0,\dots,0)$ for every nonnegative sequence $(a_i)$ with $\sum_{i=1}^n a_i = 1$. A function $\Psi\colon [0,\infty)^n \to \R$ which is symmetric (with respect to permuting the coordinates) is said to be \emph{Schur-convex} if $\Psi(a) \leq \Psi(b)$ whenever $a \prec b$ and \emph{Schur-concave} if $\Psi(a) \geq \Psi(b)$ whenever $a \prec b$. For instance, a function of the form $\Psi(a) = \sum_{i=1}^n \psi(a_i)$ with $\psi\colon [0,+\infty) \to \R$ being convex is Schur-convex.

Now we consider a stronger claim for the case with no mass at 0.

\begin{theorem}\label{thm:2-p>3}
Let $L$ be a positive integer. Let $X_1,X_2,\ldots$ be i.i.d. copies of a random variable $X$ with $\p{X = -j} = \p{X = j} = \frac{1}{2L}$, $j = 1, \ldots, L$. For every $n \geq 1$, reals $a_1,\ldots,a_n$ and $p \geq 3$, we have
\begin{equation}\label{eq:2-p>3}
\left(\E\left|\sum_{i=1}^n a_iX_i \right|^p\right)^{1/p} \leq C_p \left(\E\left|\sum_{i=1}^n a_iX_i \right|^2\right)^{1/2} 
\end{equation}
with $C_p = \sqrt{2} \Big(\frac{\Gamma (\frac{p+1}{2})}{\sqrt{\pi}} \Big)^{1/p}$ which is sharp.
\end{theorem}

And if we restrict our attention to a random sign with some mass added at 0, we can achieve a Schur convexity type result, which in turn yields khintchine type inequalities.

\begin{theorem}\label{thm:Schur}
Let $\rho_0 \in [0,\frac{1}{2}]$. Let $X_1,X_2,\ldots$ be i.i.d. copies of a random variable $X$ with $\p{X = 0} = \rho_0$ and $\p{X = -1} = \p{X = 1} = \frac{1-\rho_0}{2}$. Let $p \geq 3$. For every $n \geq 1$ and reals $a_1,\ldots,a_n, b_1, \ldots, b_n$ such that $(a_i^2)_{i=1}^n \prec (b_i^2)_{i=1}^n$, we have
\begin{equation}\label{eq:Schur}
\E\left|\sum_{i=1}^n a_iX_i \right|^p \geq \E\left|\sum_{i=1}^n b_iX_i \right|^p.
\end{equation}
\end{theorem}

\begin{corollary}
Under the assumptions of Theorem \ref{thm:Schur} for every $n \geq 1$ and reals $a_1,\ldots,a_n$, we have
\begin{equation}\label{eq:2-p>3'}
\left(\E\left|\sum_{i=1}^n a_iX_i \right|^p\right)^{1/p} \leq C_p \left(\E\left|\sum_{i=1}^n a_iX_i \right|^2\right)^{1/2} 
\end{equation}
with $C_p = \sqrt{2} \Big(\frac{\Gamma (\frac{p+1}{2})}{\sqrt{\pi}} \Big)^{1/p}$ which is sharp.
\end{corollary}

Finally we consider the $p=1$ case and find the sharp constant agreeing with a random sign.

\begin{theorem}\label{thm:L1-L2}
Let $\rho_0 \in [\frac{1}{2},1]$ and let $L$ be a positive integer. Let $X_1,X_2,\ldots$ be i.i.d. copies of a random variable $X$ with $\p{X = 0} = \rho_0$ and $\p{X = -j} = \p{X = j} = \frac{1-\rho_0}{2L}$, $j = 1, \ldots, L$. For every $n \geq 1$ and reals $a_1,\ldots,a_n$, we have
\begin{equation}\label{eq:L1-L2}
\E\left|\sum_{i=1}^n a_iX_i \right| \geq c_1\left(\E\left|\sum_{i=1}^n a_iX_i \right|^2\right)^{1/2} 
\end{equation}
with $c_1 = \frac{\E|X|}{\sqrt{\E|X|^2}} = \sqrt{\frac{3(1-\rho_0)L(L+1)}{2(2L+1)}}$ which is sharp.
\end{theorem}


\subsection{Uniform Random Variables}



\subsection{Gaussian Mixtures}

First we look at \cite{ENT}. Gaussian measures very important in convex geometry and local banach space theory. 

Recall what a gaussian mixture is:

\begin{defn}
	A random variable X is a \textit{Gaussian mixture} if there exists a positive rando variable Y and standard Gaussian Z independent from Y s.t. X has the same distribution as YZ.
\end{defn}

\begin{exmp}
	A random variable with density $f(x) = \sum_{j=1}^m p_j \frac{1}{\sqrt{2\pi}\sigma_j}e^{-x^2/2\sigma_j^2}$ is a gaussian mixture
\end{exmp}

We also recall definitions of majorization:

\begin{defn}
	$a \in \R^n$ is majorized by $b \in \R^n$, written $a \prec b$, if for $k = 1,...,n-1$ we have 
	\begin{align*}
		\sum_{j=1}^k a_j^* \leq \sum_{j=1}^k b_j^*
	\end{align*}
	,where $a^*$ is the decreasing rearrangement, and we have $\sum_{j=1}^n a_j = \sum_{j=1}^n b_j$. 
\end{defn}

\begin{defn}
	Then a function preserving $\prec$ is Schur convex
\end{defn}

This allows us to state:

\begin{theorem}
	Let X be a Gaussin mixture and $X_1,...,X_n$ be independent copies of X. For two vectors $a,b \in \R^n$ with $p \geq 2$ we have
	\begin{align*}
		(a_1^2,...,a_n^2) \prec (b_1^2,...,b_n^2) \implies ||\sum_{i=1}^n a_iX_i||_p \leq ||\sum_{i=1}^n b_iX_i||_p
	\end{align*}
\end{theorem}

%need to cover bernstein's theorem

The proof relies on Bernstein's theorem:

\begin{theorem}
	$g : (0,\infty) \to \R$ is completely monotonic, ie. $(-1)^ng^{(n)} \geq 0$, if and only if $\exists$ non-negative Borel measure $\mu$ on $[0,\infty)$ s.t.
	\begin{align*}
		f(x) = \int_0^{\infty} e^{-tx}d\mu(t)
	\end{align*}
\end{theorem}

Bernstein in turn can be used to show the following:

\begin{theorem}
	A symmeric random variable X with density f is a Gaussian mixture if and only if $x \to f(\sqrt{x})$ is completely monotonic for $x > 0$. 
\end{theorem}

%Why???

\begin{proof}
	Let X be a symmetric random variable with density f with $x \to f(\sqrt(x))$ is completely monotonic. By Bernstein there exists a non-negative Borel measure $\mu$ supported on $[0,\infty)$ s.t. $f(\sqrt{x}) = \int_0^{\infty}e^{-tx}d\mu(t)$. We can now compute
	\begin{align*}
		P(X \in A) = \int_A \int_0^{\infty} e^{-tx^2}d\mu(t)dx = \int_0^{\infty}\int_A e^{-tx^2}dxd\mu(t) = \int_0^{\infty} \int_{\sqrt{2t}A} \frac{1}{\sqrt{2\pi}}e^{-x^2/2}dx
	\end{align*}
	and then simplify to $\int_0^{\infty}\gamma_n(\sqrt{2t}A)d\nu(t)$ by writing $d\nu(t) = \sqrt{\frac{\pi}{t}}d\mu(t)$. We note this is a probability measure.

	Now let $V$ be distributed via $\nu$. V is positive and hence we can write $Y = \frac{1}{\sqrt{2V}}$. Then
	\begin{align*}
		P(YZ \in A) = P(\frac{1}{2V} Z \in A) = \int_0^{\infty}\gamma_n(\sqrt{2t}A)d\nu(t) = P(X \in A)
	\end{align*}

	which concludes the proof. The converse follows via bernstein and noting X has density $f(x) = \frac{1}{\sqrt{2 \pi}} \int_0^{\infty} e^{-x^2/2y^2}\frac{d\nu(y)}{y}$
\end{proof}

Note this is a la \cite{LO}. Similarly inspired by \cite{BGMN} we have a result for vectors uniformly distributed on $l_p$ balls:

\begin{theorem}
	Fix $q \in (0,2]$ and let $X = (X_1,...,X_n)$ a random vector uniformly distributed on $B_q^n$. For two vectors $(a_1,...,a_n)$ and $(b_1,...,b_n)$ in $\R^n$ with $p \geq 2$ we have
	\begin{align*}
		(a_1^2,...,a_n^2) \prec (b_1^2,...,b_n^2) \implies ||\sum^n a_i X_i||_p \leq ||\sum^n b_i X_i||_p
	\end{align*}

	whereas for $p \in (-1,2)$ the second inequality is reversed
\end{theorem}




\subsection{Random Vectors Uniform on $\mathbb{S}^{n-1}$}

\section{Applications}

\subsection{Functional Analysis}

The Khintchine-Kahane inequalities follow as a vector valued generalization of Khintchine:

\begin{theorem}
	Let S a Rademacher series in banach B with $\sigma = \sigma(S)$. Let $M = M(S)$ denote the median of $\norm{S}$. Then $\forall t > 0$
	\begin{align*}
		\Pp (|\norm{S}- M| > t) \leq 4 e^{-t^2/8\sigma^2}
	\end{align*}
	and in particular $\exists \alpha > 0$ s.t. $\E e^{\alpha \norm{S}^2} < \infty$ and all moments of S are equivalent, ie. for $0 < p , q < \infty$ we have $\exists C_{p,q}$ s.t.
	\begin{align*}
		\norm{S}_p \leq C_{p,q} \norm{S}_q
	\end{align*}
\end{theorem}

\subsection{Type and Cotype of Banach Spaces}

\begin{defn}
	Banach space B is of \textit{type p} if $\exists C$ s.t. $\forall $ finite sequences $(x_i)$ in B,
	\begin{align*}
		\norm{\sum_i \epsilon_ix_i}_p \leq C (\sum_i \norm{x_i}^p)^{(1/p)}
	\end{align*}
\end{defn}

%Generalzation of triangle inequality.

Every space of type 1 via triangle inequality. Only makes senes for $p \leq 2$ due to Khintchine's inequalities.%?why?

\begin{defn}
	A banach space B is of \textit{cotype q} if $\exists$ $C$ s.t. $\forall$ finite sequences $(x_i)$ in B we have
	\begin{align*}
		(\sum_i \norm{x_i}^q)^{1/q} \leq C \norm{\sum_i \epsilon_i x_i}_q
	\end{align*}
\end{defn}

Every banach space of is infinite cotype via Levy's triangle inequality arguments. Again only makes sense for $q \geq 2$ via Khintchine.

The best we can possibly do is have type 2 and cotype 2, which is achieved in hilbert spaces via orthogonality:

\begin{align*}
	\E \norm{\sum_i \epsilon e_i} = \sum_i\norm{e_i}^2
\end{align*}

\begin{theorem}
	A banach space B is of type 2 and cotype 2 $\iff$ it is isomorphic to a Hilbert space
\end{theorem}

\subsection{Convex Geometry}

See

\begin{verbatim}
	Eskenazis, A. Nayar, P. Tkocz, T. Gaussian Mixtures: Entropy and Geometric Inequalities.
\end{verbatim}



\begin{thebibliography}{9}

\bibitem{TT} Tkocz T. Khinchin Inequalities with Sharp Constants. 

\begin{verbatim}
	http://www.him.uni-bonn.de/fileadmin/him/Lecture_Notes/Khinchin_inequalities_-_slides_3.pdf
\end{verbatim}

\bibitem{H} Haagerup U. The Best Constants in the Khintchine Inequality. 

\begin{verbatim}
	https://eudml.org/doc/218383
\end{verbatim}

\bibitem{S} Szarek H. On the best constants in the Khinchin's inequality. 

\begin{verbatim}
	https://eudml.org/doc/218071
\end{verbatim}

\bibitem{HT} Havrilla. A, Tkocz. T, Sharp Khinchin-type inequalities for symmetric discrete uniform random variables. 

\begin{verbatim}
	https://arxiv.org/abs/1912.13345
\end{verbatim}

\bibitem{HNT} Havrilla .A, Nayar. P, Tkocz. T, Khinchin-type inequalities via Hadamard's factorisation. 

\begin{verbatim}
	https://arxiv.org/abs/2102.09500
\end{verbatim}

\bibitem{NO} Nayar. P, Oleszkiewicz. Khinchine type inequalties with optimal constants via ultra log-concavity. 

\begin{verbatim}
	https://link.springer.com/article/10.1007/s11117-011-0130-z
\end{verbatim}

\bibitem{KH} J.-P. Kahane, Sur les sommes vectorielles, C. R. Acad Sci. Paris 259(1964), 2577-2580

\bibitem{LT} Ledoux. M, Talagrand. M, Probability in Banach Spaces. 

\begin{verbatim}
	https://link.springer.com/book/10.1007/978-3-642-20212-4
\end{verbatim}

\bibitem{M} Mordhorst, O. The optimal constants in Khintchine's inequality for the case $2 < p < 3^*$. 

\begin{verbatim}
	https://arxiv.org/pdf/1601.07850.pdf
\end{verbatim}

\bibitem{NP} Nazarov, F. Podkorytov, A. Ball, Haagerup, and Distribution Functions. 

\begin{verbatim}
	https://link.springer.com/chapter/10.1007/978-3-0348-8378-8_21
\end{verbatim}

\bibitem{ENT} Eskenazis, A. Nayar, P. Tkocz, T. Gaussian Mixtures: Entropy and Geometric Inequalities.

\begin{verbatim}
	https://arxiv.org/pdf/1611.04921.pdf
\end{verbatim}

\bibitem{LO} Latala, R. Oleszkiewicz, K. A note on sums of independent uniformly distributed random variables.

\begin{verbatim}
	https://www.mimuw.edu.pl/~rlatala/papers/cm6826.pdf
\end{verbatim}

\bibitem{BGMN} Barthe, F. Guedon, O. Mendelson, S. Naor, A. A Probabilistic Approach To The Geometry of the $l_P^N$-Ball.

\begin{verbatim}
	https://arxiv.org/pdf/math/0503650.pdf
\end{verbatim}

\bibitem{S} Szarek, S. On the best constants in the Khinchin inequality.

\begin{verbatim}
	https://eudml.org/doc/218071
\end{verbatim}

\bibitem{Y} Young, R. On the Best Possible Constants in the Khintchine Inequality.

\begin{verbatim}
	https://academic.oup.com/jlms/article/s2-14/3/496/2939431?login=true
\end{verbatim}

\bibitem{N} Newman, An Extension of Khintchines Inequality. 

\begin{verbatim}
	https://projecteuclid.org/journals/bulletin-of-the-american-mathematical-society-new-series/volume-81/issue-5/An-extension-of-Khintchines-inequality/bams/1183537247.full
\end{verbatim}

\end{thebibliography}

\section{Appendix}

\subsection{Examples of Type L Random Variables}

We list some examples of probability distributions of type L. In what follows, $X$ is a symmetric random variable.

\begin{enumerate}[(a)]

\item 
Let $X$ be integer-valued with $\p{X = 0} = p_0$ and $\p{X = -k} = \p{X = k} = p_k$, $k = 1, \dots, n$ for nonnegative $p_0, \dots, p_n$ with $p_0 + 2\sum_{k=1}^n p_k = 1$. 

If $\frac{1}{2}p_0 \leq p_1 \leq \dots \leq p_n$, then $\E \cos(zX) = p_0 + \sum_{k=1}^n (2p_k)\cos(kz)$ has only real zeros, as it follows from the Enestr\"om-Kakeya theorem (see, e.g. Problem III.204 in \cite{PS1}). As a result, $X$ is of type $\sL$. In particular, if $X$ is uniform on $\{-n,\dots, 1, 1, \dots, n\}$ with a possible atom at $0$ satisfying $\p{X = 0} \leq \frac{1}{n+1}$, then $X$ is of type L

By the symmetry of $X$, the polynomial $Q(w) = \E w^{X+n}$ is self-inversive (the sequence of its coefficients is a palindrome, in other words, $w^{2n}Q(1/w) = Q(w)$). In particular, all its roots are symmetric with respect to the unit circle, that is if $w_0$ is a root of $Q$, then so is $1/w_0$. For instance, if for some $\alpha \geq 1$,
\[
\frac{1}{2}p_0^\alpha + \sum_{k=1}^{n-1} p_k^\alpha \leq \left(\frac{2}{n-2}\right)^{\alpha-1}p_n^\alpha,
\]
where $n$ is the number of nonzero coefficients of $Q$, then $Q$ has zeros only on the unit circle, so $X$ is of type L

\item
Let $X$ take values in $[-1,1]$ and have a density $f$ (which is even). Each of the following conditions implies that $X$ is of type L.
\begin{enumerate}[(i)]
\item $f$ is  nondecreasing on $(0,1)$.

\item $f$ is $C^2$ with $f' < 0$ and $f'' < 0$ on $(0,1)$.

\item $f(t) = h(t)^{\alpha}$, where $\alpha > -1$ and $h$ is an entire even function which is real-valued on the real line, $h(1) = 0$, $h(0) > 0$ and $h'(iz)$ is  in the Laguerre-P\'olya class. In particular, $f(t) = \text{const}\cdot (1-t^{2m})^{\alpha}$ for a nonnegative integer $m$.  
\end{enumerate}

Moreover, if $X$ has a density on $\R$ of the following form, then it is of type L.

\begin{enumerate}
\item[(iv)] $f(t) = \text{const}\cdot e^{-t^{2m}}$ for a nonnegative integer $m$.  

\item[(v)] $f(t) = (2\pi)^{-1/2}e^{-t^2/2}(1-b+ bt^2)$, $0 \leq b \leq 1$.
\end{enumerate}


\end{enumerate}




\end{document}



