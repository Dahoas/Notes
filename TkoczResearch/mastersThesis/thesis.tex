\documentclass[10pt]{article}


\usepackage{amssymb,amsthm,amsmath}
\usepackage{enumerate}
\usepackage{graphicx,color}
\usepackage[hidelinks]{hyperref}
%\usepackage{refcheck}

\newcommand{\dd}{\mathrm{d}}
\newcommand{\E}{\mathbb{E}}
\newcommand{\1}{\textbf{1}}
\newcommand{\R}{\mathbb{R}}
\newcommand{\C}{\mathbb{C}} 
\newcommand{\Z}{\mathbb{Z}}
\newcommand{\N}{\mathbb{N}}
\newcommand{\p}[1]{\mathbb{P}\left( #1 \right)}
\newcommand{\scal}[2]{\left\langle #1, #2 \right\rangle}
\newcommand{\red}{\color{red}}
\newcommand{\shift}{\vdash}

\DeclareMathOperator{\Var}{Var}
\DeclareMathOperator{\sgn}{sgn}

\usepackage[paper=a4paper, left=1.3in, right=1.3in, top=1in, bottom=1in]{geometry}
\linespread{1.3}
\pagestyle{plain}

\newtheorem{theorem}{Theorem}
\newtheorem{lemma}[theorem]{Lemma}
\newtheorem{corollary}[theorem]{Corollary}

\theoremstyle{remark}
\newtheorem{remark}[theorem]{Remark}


\newtheorem{conjecture}{Conjecture}

\theoremstyle{definition}
\newtheorem{defn}[theorem]{Definition}
\newtheorem{exmp}[theorem]{Example}


\title{\vspace{-3em}A Study of Khintchine Type Inequalities for Random Variables}
\author{Alex Havrilla}



\begin{document}

\maketitle

\section{An Overview of Sharp Khintchine Type Inequalities}

\section{Khintchine Type Inequalities for Classes of Random Variables}

\begin{itemize}
	\item General Form
\end{itemize}

\subsection{Random Signs}

\begin{itemize}
	\item Definition
	\item Haagerup's Approach
	\item Variations: Dependent random signs.
\end{itemize}

file:///C:/Users/Alex/Downloads/Haagerup%20-%20The%20Best%20Constants%20in%20the%20khinchin%20inequality%20(1).pdf

\subsection{The Classical Vector Setting}

\begin{itemize}
	\item Definitions
	\item Khintchine Kahane
\end{itemize}

\subsection{Ultra Sub Gaussian Random Variables}

\begin{itemize}
	\item Definitions
	\item Log Concavity
	\item Khintchine Results
	\item Examples
	\item Connection to Strong Concavity
\end{itemize}

\subsection{Type L Random Variables}
	
\begin{itemize}
	\item Definitions
	\item Khintchine Results
	\item Examples: Polya, Newman, Renyi etc
	\item Connection to Strong Concavity and Ultra Sub Gaussianity
\end{itemize}

\subsection{Generalized Random Signs}

\begin{itemize}
	\item Definitions
	\item Khintchine Results
	\item Connection to Type L
\end{itemize}

https://www.math.cmu.edu/~ttkocz/mypapers/mathematics/khinchin-disc-unif.pdf

\subsection{Vectors on Spheres}

\begin{itemize}
	\item Results in the Uniform Case
	\item Majorization
	\item Bisubharmonicity
	\item Uniform Result on Sphere for Bisubharmonic Functions
	\item Results with Rotational Invariance
\end{itemize}

https://www.researchgate.net/publication/226779758_Best_Khintchine_Type_Inequalities_for_Sums_of_Independent_Rotationally_Invariant_Random_Vectors

https://www.mimuw.edu.pl/~rlatala/papers/cm6826.pdf

https://pdfs.semanticscholar.org/30af/d8b3a15c3b180e94f8d39c3cdfd90540bba5.pdf

\subsection{Gaussian Mixtures}

\begin{itemize}
	\item Mixtures
	\item Results for Gaussian Mixtures
\end{itemize}

https://arxiv.org/pdf/1611.04921.pdf

\subsection{The Exponential Family}

\begin{itemize}
	\item Definitions
	\item Results for Exponential Family
\end{itemize}

https://arxiv.org/abs/1801.07597

\section{Applications}

\begin{itemize}
	\item Littlewood-Payley Decomposition
	\item Banach Space Type
	\item Grothendieck Inequality
	\item Projections of Cross-Polytopes
\end{itemize}

Banach space: absolute convergence of series. 
	Reimann's theorem: series is unconditinoal if absolute. But if not absolutely converging then not unconditionally converging

Banach: Permutation invariant, iff convering absolutely: If + and - is convering: Classical absolute convergene is too strong



\begin{thebibliography}{9}

https://arxiv.org/pdf/1611.04921.pdf

https://www.researchgate.net/publication/226779758_Best_Khintchine_Type_Inequalities_for_Sums_of_Independent_Rotationally_Invariant_Random_Vectors

https://pdfs.semanticscholar.org/30af/d8b3a15c3b180e94f8d39c3cdfd90540bba5.pdf

https://www.math.cmu.edu/~ttkocz/mypapers/mathematics/khinchin-disc-unif.pdf

http://math.iisc.ac.in/~khare/papers/khinchin-kahane.pdf

https://www.mimuw.edu.pl/~rlatala/papers/cm6826.pdf

file:///C:/Users/Alex/Downloads/Haagerup%20-%20The%20Best%20Constants%20in%20the%20khinchin%20inequality%20(1).pdf

https://arxiv.org/abs/1801.07597

%Tkocz lecture at HIM 
http://www.him.uni-bonn.de/fileadmin/him/Lecture_Notes/Khinchin_inequalities_-_slides_3.pdf

\bibitem{H} Haagerup U. The Best Constants in the Khintchine Inequality. file:///C:/Users/Alex/Downloads/haagerup%20khintchine.pdf

\bibitem{HT} Havrilla. A, Tkocz. T, Sharp Khinchin-type inequalities for symmetric discrete uniform random variables. https://arxiv.org/abs/1912.13345

\bibitem{LT} Isoperimetry and Process file:///C:/Users/Alex/Downloads/ledoux%20talagrand%20-%20probab%20in%20banach%20spaces.pdf

\bibitem{NO} Nayar, P. Oleskiewiz. Khintchin-type Inequalities with optimal constants via Ultra-Log Concavity


\bibitem{MR} Markovsky. I, Shodhan. R,
Palindromic Polynomials, Time-Reversible Systems, and Conserved Quantities. https://eprints.soton.ac.uk/266592/1/Med08.pdf

\bibitem{KB} Keel. L, Bhattarcharyya. S,
A New Proof of the Jury Test. https://ieeexplore.ieee.org/document/703305
%\bibitem{Z-2} 
%https://mathoverflow.net/questions/208349
\bibitem{NC} Newman, An Extension of Khintchines Inequality. 

\bibitem{PS} Polya G. Szego G. Problems and Theorems in Analysis. 

\bibitem{HP} Hallum P. Zeroes of Entire Functions Represented By Fourier Transforms
%20-%20An%20extension%20of%20the%20khinchin%20inequality%20(3).pdf

\bibitem{NW} Newman C. Wu W. Lee Yang Property and Gaussian Multiplicative Chaos

\bibitem{DR} Dimitrov D. Rusev P. Zeroes of Entire Fourier Transforms

\end{thebibliography}

\end{document}



