\documentclass[10pt]{article}


\usepackage{amssymb,amsthm,amsmath}
\usepackage{enumerate}
\usepackage{graphicx,color}
\usepackage[hidelinks]{hyperref}
%\usepackage{refcheck}

\newcommand{\dd}{\mathrm{d}}
\newcommand{\E}{\mathbb{E}}
\newcommand{\1}{\textbf{1}}
\newcommand{\R}{\mathbb{R}}
\newcommand{\C}{\mathbb{C}} 
\newcommand{\Z}{\mathbb{Z}}
\newcommand{\N}{\mathbb{N}}
\newcommand{\p}[1]{\mathbb{P}\left( #1 \right)}
\newcommand{\scal}[2]{\left\langle #1, #2 \right\rangle}
\newcommand{\red}{\color{red}}
\newcommand{\shift}{\vdash}

\DeclareMathOperator{\Var}{Var}
\DeclareMathOperator{\sgn}{sgn}

\usepackage[paper=a4paper, left=1.3in, right=1.3in, top=1in, bottom=1in]{geometry}
\linespread{1.3}
\pagestyle{plain}

\newtheorem{theorem}{Theorem}
\newtheorem{lemma}[theorem]{Lemma}
\newtheorem{corollary}[theorem]{Corollary}

\theoremstyle{remark}
\newtheorem{remark}[theorem]{Remark}


\newtheorem{conjecture}{Conjecture}

\theoremstyle{definition}
\newtheorem{defn}[theorem]{Definition}
\newtheorem{exmp}[theorem]{Example}


\title{\vspace{-3em}A Study of Khintchine Type Inequalities for Random Variables}
\author{Alex Havrilla}



\begin{document}

\maketitle

\section{An Overview of Sharp Khintchine Type Inequalities}

\section{Khintchine Type Inequalities for Classes of Random Variables}

\begin{itemize}
	\item General Form
\end{itemize}

\subsection{Random Signs}

\begin{itemize}
	\item Definition
	\item Haagerup's Approach
	\item Variations: Dependent random signs.
\end{itemize}

file:///C:/Users/Alex/Downloads/Haagerup%20-%20The%20Best%20Constants%20in%20the%20khinchin%20inequality%20(1).pdf

\subsection{The Classical Vector Setting}

\begin{itemize}
	\item Definitions
	\item Khintchine Kahane
\end{itemize}

\subsection{Ultra Sub Gaussian Random Variables}

\begin{itemize}
	\item Definitions
	\item Log Concavity
	\item Khintchine Results
	\item Examples
	\item Connection to Strong Concavity
\end{itemize}

\subsection{Type L Random Variables}
	
\begin{itemize}
	\item Definitions
	\item Khintchine Results
	\item Examples: Polya, Newman, Renyi etc
	\item Connection to Strong Concavity and Ultra Sub Gaussianity
\end{itemize}

\subsection{Generalized Random Signs}

\begin{itemize}
	\item Definitions
	\item Khintchine Results
	\item Connection to Type L
\end{itemize}

https://www.math.cmu.edu/~ttkocz/mypapers/mathematics/khinchin-disc-unif.pdf

\subsection{Vectors on Spheres}

\begin{itemize}
	\item Results in the Uniform Case
	\item Majorization
	\item Bisubharmonicity
	\item Uniform Result on Sphere for Bisubharmonic Functions
	\item Results with Rotational Invariance
\end{itemize}

https://www.researchgate.net/publication/226779758_Best_Khintchine_Type_Inequalities_for_Sums_of_Independent_Rotationally_Invariant_Random_Vectors

https://www.mimuw.edu.pl/~rlatala/papers/cm6826.pdf

https://pdfs.semanticscholar.org/30af/d8b3a15c3b180e94f8d39c3cdfd90540bba5.pdf

\subsection{Gaussian Mixtures}

\begin{itemize}
	\item Mixtures
	\item Results for Gaussian Mixtures
\end{itemize}

https://arxiv.org/pdf/1611.04921.pdf

\subsection{The Exponential Family}

\begin{itemize}
	\item Definitions
	\item Results for Exponential Family
\end{itemize}

https://arxiv.org/abs/1801.07597

\section{Applications}

\begin{itemize}
	\item Littlewood-Payley Decomposition
	\item Banach Space Type
	\item Grothendieck Inequality
	\item Projections of Cross-Polytopes
\end{itemize}

Banach space: absolute convergence of series. 
	Reimann's theorem: series is unconditinoal if absolute. But if not absolutely converging then not unconditionally converging

Banach: Permutation invariant, iff convering absolutely: If + and - is convering: Classical absolute convergene is too strong

\section{Notes}

\subsection{Ultra Sub Gaussanity}

\begin{verbatim}
	https://link.springer.com/content/pdf/10.1007/s11117-011-0130-z.pdf
\end{verbatim}

\begin{remark}
	Haagerup finding best constants for $p \in (2,3)$ and $q$ in $(0,2)$ resolved parts important in applictaions since 2 norm easily computable
\end{remark}

\begin{remark}
	Problem of finding optimal constants for general $C_{p,q}$ still open. Note $C_{p,q}$ is for $(\E|S|^p)^{1/p} \leq C_{p,q} (\E|S|^q)^{1/q}$
\end{remark}

\begin{remark}
	Cases:
	\begin{enumerate}
		\item even p,q and p divisible by q: \begin{verbatim}Czerwi  ́nski, W.: Khinchine inequalities (in Polish). University of Warsaw, Master thesis (2008)\end{verbatim}
		\item Replace random signs by rotationally invariant special density: \begin{verbatim} ajorization of sequences, sharp vector Khinchin inequalities,and bisubharmonic functions. Stud. Math \end{verbatim} and \begin{verbatim} Best Khintchine type inequalities for sums of independent, rotationally invari-ant random vectors \end{verbatim}
	\end{enumerate}
\end{remark}

\begin{remark}
	Main tool to prove is walkups theorem
\end{remark}

\begin{remark}
	Note: on \begin{verbatim} http://www.him.uni-bonn.de/fileadmin/him/Lecture_Notes/Khinchin_inequalities_-_slides_3.pdf \end{verbatim}
\end{remark}

\subsection{Gaussian Mixtures}

\begin{verbatim}
	https://arxiv.org/pdf/1611.04921.pdf
\end{verbatim}

\section{Papers to Investigate}

See page 2 from 

\begin{verbatim}
	https://arxiv.org/pdf/1611.04921.pdf
\end{verbatim}

Latala and Oleszkiewicz show khintchine via schur concavity of some function of uniform random variables.

\section{Questions}

Should I include or omit some proofs?

\section{Holding}

\subsection{Type L Random Variables}

Relatively few examples of Type L random variables are known. The majority we do have follow from results of Polya in his study of kernels producing strictly real zeroes of fourier transforms of the form:

\begin{align*}
	\phi(z) = \int_{\R} K(x)cos(zx)dx
\end{align*}

for some kernel $K : \R \to \R$. This can naturally be interpreted as the inverse fourier transform of a random variable $X$ with density $K$. And (assuming symmetry and nice gaussanity conditions) if $\phi$ has strictly real zeroes then we know $\phi(iz) = \E e^{-zX}$ has strictly imaginary zeroes and hence X is type L.

\subsubsection{Examples of Type L Random Variables}

\textbf{Polya's Examples:}

All of these examples can be found in Polya's \textit{Problems in Analysis} but the experience of retrieving them(and their proofs) is somewhat time consuming. Hopefully this presentation is somewhat less so. We attach proofs of these examples in an appendix. 

\begin{theorem}[Decreasing Concave Density(173)] 
	Let $X$ be a symmetric continuous random variable distributed on $[0,1]$ density $f$ s.t. $f',f'' < 0$. Then $X \in \lL$.
\end{theorem}

\begin{theorem}[L1 Bounded Derivative(175)] 
	Let $X$ be a symmetric continuous random variable distributed on $[0,1]$ with density f s.t. $|f(1)| \geq \int_0^1 |f'(t)|dt$. Then $X \in \lL$. Note in particular this works for the case $f$ is increasing.
\end{theorem}

\begin{theorem}[Exponential Density(170)] 
	Let $\alpha$ be even integer greater than 2. Then if $X$ a symmetric continuous random variable with density of the form $e^{-t^{\alpha}}$ then $X \in \lL$
\end{theorem}

\begin{theorem}[Exponential Product Density(161)]
	Let $1 > \alpha \geq 0, 0 < \alpha_1 \leq \alpha_2 \leq ...$ and reciprocal convergent. Then if $g(z)= e^{-\alpha z} (1-\frac{z}{\alpha_1})(1-\frac{z}{\alpha_2})...$ we have for symmetric X with density $e^{-t^2} g(-t^2)$ then $X \in \lL$.
\end{theorem}

\begin{theorem}[Bessel Function(159)] 
	The symmetric continuous random variable X with density $\frac{2}{\pi\sqrt{1-t^2}}$ in $\lL$.
\end{theorem} 

\begin{theorem}[Large nth Coefficient(27)] 
	Suppose X a discrete integer valued symmetric distribution. If $p_0 + 2p_1 +... + 2p_{n-1} < 2p_n$ then $X \in \lL$. 
\end{theorem}

%Am I missing one of the examples Tkocz went through? With increasing density or whatever?

\textbf{Newman's Examples}

Newman, who initiated our study in Type L random variables, produced some examples as well. 

Perhaps one of the most basic examples of type L random variables are arithmetic progressions and uniform random variables.

\begin{theorem}[Arithmetic Sequences] 
	Let the sequence X above be an arbitrary arithmetic progression, ie. of the form $x_1 = d$, $x_2 = d+c$,..., $x_L = d + (L-1)c$ for arbitrary $d \in \R, c > 0$. Then $S_X(z)$ has zeroes only on the imaginary axis.
\end{theorem}

\begin{theorem}[Uniform(Newman 7)] 
	Let $X$ be random variable with density $\frac{d\mu}{dy} = 1$ if $|y| \leq A$ and 0 otherwise. $A > 0$. Then $X \in \lL$.
\end{theorem}

We also know marginals of uniformly random vectors on spheres are types L.

\begin{theorem}[Newman (8)] 
	Density $(1-y^2)^{(d-2)/2}$ with $|y| \leq 1$ and 0 otherwise. For $d > 0$. 
\end{theorem}

\begin{theorem}[Newman (9)] 
	Density $e^{-\lambda cosh(y)}$, $\lambda >0$
\end{theorem}

(Some physics field theory context)

\begin{theorem}[Newman (10)] 
	$e^{-ay^4-by^2}$ with $a>0$
\end{theorem}

\textbf{Other Examples:}

\begin{theorem}[Enestrom-Kakeya] 
	If $X$ integer valued symmetric with $0\leq p_0 \leq 2 p_1 \leq ... \leq 2p_n$ with $p_n >0$ then $X \in \lL$.
\end{theorem}

\begin{theorem}[Absolute Value] 
	Let $a_0,a_1,...,a_n \in \R$ with $|a_0| + .. | a_{n-1}| \leq |a_n|$ then the trig polys $p_c(z) = \sum_{k=0}^n a_k cos(kz)$ and the sin one have only real zeroes
\end{theorem}

\textbf{Our Examples:}

\begin{theorem}[Rapidly Decreasing Polynomial] For every $a > 0$ and $b_1,b_2,... \geq 0$ with $\sum b_j \leq a$ the function 
\begin{align*}
	e^{-at^2/2} \prod_j (1-b_jt^2)
\end{align*} 

is the characteristic function of a type L random variable.
\end{theorem}

\begin{proof}
	Let $X_1,X_2,...$ iid copies of $Z_1$ and $Z_0$ an independent standard gaussian. Then $\sum \sqrt{b_j} X_j + \sqrt{a - \sum b_j} Z_0$ is the desired random variable.
\end{proof}

\begin{theorem}[General Symmetrization] \label{GSYM}
	Suppose $X -c$ is type L for some $c \in \lL$. Then $\epsilon X \in \lL$.
\end{theorem}



\begin{thebibliography}{9}

https://arxiv.org/pdf/1611.04921.pdf

https://www.researchgate.net/publication/226779758_Best_Khintchine_Type_Inequalities_for_Sums_of_Independent_Rotationally_Invariant_Random_Vectors

https://pdfs.semanticscholar.org/30af/d8b3a15c3b180e94f8d39c3cdfd90540bba5.pdf

https://www.math.cmu.edu/~ttkocz/mypapers/mathematics/khinchin-disc-unif.pdf

http://math.iisc.ac.in/~khare/papers/khinchin-kahane.pdf

https://www.mimuw.edu.pl/~rlatala/papers/cm6826.pdf

file:///C:/Users/Alex/Downloads/Haagerup%20-%20The%20Best%20Constants%20in%20the%20khinchin%20inequality%20(1).pdf

https://arxiv.org/abs/1801.07597

%Tkocz lecture at HIM 
http://www.him.uni-bonn.de/fileadmin/him/Lecture_Notes/Khinchin_inequalities_-_slides_3.pdf

\bibitem{H} Haagerup U. The Best Constants in the Khintchine Inequality. file:///C:/Users/Alex/Downloads/haagerup%20khintchine.pdf

\bibitem{HT} Havrilla. A, Tkocz. T, Sharp Khinchin-type inequalities for symmetric discrete uniform random variables. https://arxiv.org/abs/1912.13345

\bibitem{LT} Isoperimetry and Process file:///C:/Users/Alex/Downloads/ledoux%20talagrand%20-%20probab%20in%20banach%20spaces.pdf

\bibitem{NO} Nayar, P. Oleskiewiz. Khintchin-type Inequalities with optimal constants via Ultra-Log Concavity


\bibitem{MR} Markovsky. I, Shodhan. R,
Palindromic Polynomials, Time-Reversible Systems, and Conserved Quantities. https://eprints.soton.ac.uk/266592/1/Med08.pdf

\bibitem{KB} Keel. L, Bhattarcharyya. S,
A New Proof of the Jury Test. https://ieeexplore.ieee.org/document/703305
%\bibitem{Z-2} 
%https://mathoverflow.net/questions/208349
\bibitem{NC} Newman, An Extension of Khintchines Inequality. 

\bibitem{PS} Polya G. Szego G. Problems and Theorems in Analysis. 

\bibitem{HP} Hallum P. Zeroes of Entire Functions Represented By Fourier Transforms
%20-%20An%20extension%20of%20the%20khinchin%20inequality%20(3).pdf

\bibitem{NW} Newman C. Wu W. Lee Yang Property and Gaussian Multiplicative Chaos

\bibitem{DR} Dimitrov D. Rusev P. Zeroes of Entire Fourier Transforms

\end{thebibliography}

\end{document}



