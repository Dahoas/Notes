\documentclass[11pt]{article}
%you can look for fun LaTeX packages to use hereasdf

\usepackage{amsmath}
\usepackage{amssymb}
\usepackage{fancyhdr}
\usepackage{amsthm}

\usepackage{graphicx}
\usepackage{dcolumn}
\usepackage{bm}

%fun commands for fun sets
%make sure to use these in math mode
\newcommand{\Z}{\mathbb{Z}}
\newcommand{\R}{\mathbb{R}}
\newcommand{\N}{\mathbb{N}}
\newcommand{\C}{\mathbb{C}}
\newcommand{\m}{\mathcal{M}}
\newcommand{\Tt}{\mathcal{T}}
\newcommand{\pa}{\partial}
\newcommand{\dD}{\mathcal{D}}
\newcommand{\E}{\mathbb{E}}



\oddsidemargin0cm
\topmargin-2cm    
\textwidth16.5cm   
\textheight23.5cm  

\newcommand{\question}[2] {\vspace{.25in} \hrule\vspace{0.5em}
\noindent{\bf #1: #2} \vspace{0.5em}
\hrule \vspace{.10in}}
\renewcommand{\part}[1] {\vspace{.10in} {\bf (#1)}}

\newcommand{\myname}{Alex Havrilla}
\newcommand{\myandrew}{alumhavr}
\newcommand{\myhwnum}{Hw 1}

\newtheorem{prop}{Prop}
\newtheorem{lemma}{Lemma}
\newtheorem{theorem}{Theorem}
\theoremstyle{remark}
\newtheorem*{rem}{Remark}
\newtheorem*{defi}{Def}
\newtheorem*{apps}{Application}
\newtheorem*{quest}{Question}
\newtheorem*{ans}{Answer}
\newtheorem*{interest}{Interesting}
\newtheorem*{theme}{Theme}
\newtheorem*{back}{Background}

\setlength{\parindent}{0pt}
\setlength{\parskip}{5pt plus 1pt}
 
\pagestyle{fancyplain}
\lhead{\fancyplain{}{\textbf{HW\myhwnum}}}      % Note the different brackets!
\rhead{\fancyplain{}{\myname\\ \myandrew}}
\chead{\fancyplain{}{\mycourse}}

\title{Research Log}

\linespread{1.3}

\begin{document}



\maketitle

\section{12/26}

\subsection{Inequalities}

\section{12/29}

\subsection{Lambert Function Inequalities}

\begin{verbatim}
	/home/alex/Desktop/Notes/Winter 2020/LatexForms/lambertinequalities.pdf
\end{verbatim}

\section{1/7}

I continue trying to find good inequalities for lambert.

\subsection{Lambert}

\begin{defi}
	Pochamer symbol is rising factorial
\end{defi}

\subsection{Tkoczs Algebraic Approach: type-L-bimod}

\begin{rem}
	Suprising $\E|\sum \sqrt{a_j} X_j|^p - \E |X_1|^p = C_p' P(q)$ is a polynomial
\end{rem}

\begin{quest}
	Does the supremal gaussian case work out to a polynomial?
\end{quest}

\begin{ans}
	Yes this is the last part of the writeup.
\end{ans}

\begin{quest}
	How do nonnegative coefficnets reduce to p=2 case?
\end{quest}

\begin{ans}
	test
\end{ans}

\begin{quest}
	What is $q^{(0)}$?
\end{quest}
\begin{ans}
	1 and $q^{(1)} = q$
\end{ans}

\section{1/13}

\subsection{Easy Regime}

\begin{remark}
	If we require sum of coefficients $c=1$ we can always write type L RV as sum of quadratic components(assuming it has appropriate gaussian weight).
\end{remark}

\begin{quest}
	For quadratic case validity as a RV for some characteristic is determined by $0 < b < 1$ on $(1-bt^2)$. Are there conditions for higher even orders powers? Example of type L rv which takes this form but cannot be written as sum of quadratics?
\end{quest}

\begin{quest}
	Why does tkocz say we require coefficients to sum to 1 for type L?
\end{quest}

\end{document}

