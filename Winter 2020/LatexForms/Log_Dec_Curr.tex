\documentclass[11pt]{article}
%you can look for fun LaTeX packages to use hereasdf

\usepackage{amsmath}
\usepackage{amssymb}
\usepackage{fancyhdr}
\usepackage{amsthm}

\usepackage{graphicx}
\usepackage{dcolumn}
\usepackage{bm}

%fun commands for fun sets
%make sure to use these in math mode
\newcommand{\Z}{\mathbb{Z}}
\newcommand{\R}{\mathbb{R}}
\newcommand{\N}{\mathbb{N}}
\newcommand{\C}{\mathbb{C}}
\newcommand{\m}{\mathcal{M}}
\newcommand{\Tt}{\mathcal{T}}
\newcommand{\pa}{\partial}
\newcommand{\dD}{\mathcal{D}}
\newcommand{\E}{\mathbb{E}}



\oddsidemargin0cm
\topmargin-2cm    
\textwidth16.5cm   
\textheight23.5cm  

\newcommand{\question}[2] {\vspace{.25in} \hrule\vspace{0.5em}
\noindent{\bf #1: #2} \vspace{0.5em}
\hrule \vspace{.10in}}
\renewcommand{\part}[1] {\vspace{.10in} {\bf (#1)}}

\newcommand{\myname}{Alex Havrilla}
\newcommand{\myandrew}{alumhavr}
\newcommand{\myhwnum}{Hw 1}

\newtheorem{theorem}{Theorem}
\newtheorem{prop}{Prop}
\theoremstyle{remark}
\newtheorem{lemma}{Lemma}
\newtheorem{remark}{Remark}
\newtheorem{defi}{Def}
\newtheorem{apps}{Application}
\newtheorem{quest}{Question}
\newtheorem{ans}{Answer}
\newtheorem{interest}{Interesting}
\newtheorem{theme}{Theme}
\newtheorem{back}{Background}
\newtheorem{idea}{Idea}

\setlength{\parindent}{0pt}
\setlength{\parskip}{5pt plus 1pt}
 
\pagestyle{fancyplain}
\lhead{\fancyplain{}{\textbf{HW\myhwnum}}}      % Note the different brackets!
\rhead{\fancyplain{}{\myname\\ \myandrew}}
\chead{\fancyplain{}{\mycourse}}

\linespread{1.3}

\title{Log : 10/18 - Current}

\begin{document}

\maketitle

\section{10/18}

\subsection{Differential Geometry}

\begin{verbatim}
	https://en.wikipedia.org/wiki/Riemannian_connection_on_a_surface
\end{verbatim}

\section{12/21}

\subsection{Automated Theorem Proving}

\subsubsection{Open Logic}

\begin{verbatim}
	http://builds.openlogicproject.org/
\end{verbatim}

Covers set theory, modal logic, model theory, computation, intuitionist models


\subsubsection{Lean}

\begin{verbatim}
	https://github.com/leanprover-community/mathematics_in_lean
\end{verbatim}

Tutorial: 

\begin{verbatim}
	https://github.com/leanprover-community/mathematics_in_lean
\end{verbatim}

\begin{back}
	Formal language setting: Dependent type theory
\end{back}

\begin{quest}
	What does $\leftarrow$ do
\end{quest}
\begin{ans}
	Applies reverse rule(elimation rule). Like $\leftarrow \text{mul\_assoc } $  looks for a + (c + d) to turn into a + c +d instead of other way 
\end{ans}

\begin{remark}
	Arguments to tactics are curried
\end{remark}

\begin{theme}
	Making mathematics more empirical/feedback oriented! Especially brilliant because it mirrors coding process so effectively.
\end{theme}

\begin{remark}
	Can rewrite any statement, assumption or goal
\end{remark}

\begin{prop}
	apply tactic matches conclusion of theorem to goal and makes hypotheses new goal
\end{prop}

\begin{prop}
	exact tactic finishes proof with full apply(if given proof matches goal exactly)
\end{prop}

\begin{remark}
	For working backwards
\end{remark}

\begin{quest}
	Not really sure of difference between apply and exact
\end{quest}

\section{12/29}

\subsection{Deep RL}

\begin{verbatim}
	https://cmudeeprl.github.io/Spring202010403website/lectures/
\end{verbatim}

\begin{idea}
	Using reinforcement learning in automated proof theory.
\end{idea}

\begin{prop}
	In RL often cannot use gradient optimization, in contrast to supervised learning. So instead we use non-gradient optimization methods and gradient estimators
\end{prop}

\begin{prop}

	\begin{verbatim}

		"it is comparatively easy to make computers 
		exhibit adult level performance on intelligence tests or
		 playing checkers, and difficult or impossible to give them the skills
		  of a one-year-old when it comes to perception and mobility"
	\end{verbatim}
	Hans Moravec
\end{prop}

\section{12/31}

\subsection{Northwestern Reserach}

\textbf{Transport Model for Feature Extraction}

\begin{verbatim}
	https://arxiv.org/pdf/1910.14543.pdf
\end{verbatim}

\begin{remark}
	Well known techniques for (nonlinear) feature extraction:
	\begin{itemize}
		\item Kernel PCA
		\item isomap
		\item locally linear embeddings
		\item laplacian eigenmaps
	\end{itemize}
\end{remark}

\begin{defi}
	Transport operator
	\begin{align*}
		Ty = Ly - div(vy)
	\end{align*}
	for some anti-symmetric matrix v and laplacian L
\end{defi}

\textbf{On The Energy Landscape of Spherical Spin Glass : p-Spin}

\begin{verbatim}
	https://arxiv.org/pdf/1702.08906.pdf
\end{verbatim}



\subsection{Tensor Contraction}

\begin{verbatim}
	https://www.quora.com/What-is
	-Tensor-contraction-
	How-to-compute-tensor-contraction
\end{verbatim}

\section{1/10}

\subsection{Set}

\begin{remark}
	Raw practice has allowed me to find sets quickly subconciously. Incoporating an algorithm to smooth out bias would be helpful.

	For example I first always look for homochromatic sets. This induces a bias that I am then slow to fix. Explicitly looking for other pairs would help.

	Cultivate useful habits(mental intuitions) and prune others.

	Still important to identify characteristics with surplus

	Good illutstration of the power of shifting perspectives and the necessity/efficiency of raw practice/intuition

	Methods for identifying all diff: identify bottlenecks and work from those. Or raw search

	Concious overhead corrupts subconciuos pattern recoginition(efficiency).
\end{remark}

\subsection{Chess}

\begin{remark}
	Don't make silly mistakes
\end{remark}

\section{1/11}

\subsection{Variational Techniques in Stochastic Geometry}

\begin{remark}
	Looking at random variables on graphs
\end{remark}

\begin{defi}
	U-stat : something nice
\end{defi}

\begin{defi}
	$1_{0 \to_{Z({\alpha})}} \partial B_n(0)}$ is even we can raverse to boundary only on boolean occupied area
\end{defi}

\begin{remark}
	Goal is to control second order properties(variances). 
\end{remark}

\begin{defi}
	Mecke dormula encompes results for poisson process
\end{defi}

\begin{remark}
	OU semigroup: behaves differently on poisson vs. gaussian vs. hypercube measures. Is non hypercontractice ie. no log-sobolev
\end{remark}

\begin{remark}
	Charlize polynomials are orthogonal under poisson density
\end{remark}

\subsection{Chess}

\begin{remark}
	Want to compute faster somehow. Spend more time computign when it's not my turn. To compute efficiently think ADVERSARIALLY(what does my opponent want?)
\end{remark}

\begin{remark}
	Can't be tunnel visiond.
\end{remark}

\begin{remark}
	Don't mentally slack when ahead. Be ruthless
\end{remark}

\end{document}

