\documentclass[11pt]{article}
%you can look for fun LaTeX packages to use hereasdf

\usepackage{amsmath}
\usepackage{amssymb}
\usepackage{fancyhdr}
\usepackage{amsthm}

\usepackage{graphicx}
\usepackage{dcolumn}
\usepackage{bm}

%fun commands for fun sets
%make sure to use these in math mode
\newcommand{\Z}{\mathbb{Z}}
\newcommand{\R}{\mathbb{R}}
\newcommand{\N}{\mathbb{N}}
\newcommand{\C}{\mathbb{C}}
\newcommand{\m}{\mathcal{M}}
\newcommand{\Tt}{\mathcal{T}}
\newcommand{\pa}{\partial}
\newcommand{\dD}{\mathcal{D}}
\newcommand{\E}{\mathbb{E}}



\oddsidemargin0cm
\topmargin-2cm    
\textwidth16.5cm   
\textheight23.5cm  

\newcommand{\question}[2] {\vspace{.25in} \hrule\vspace{0.5em}
\noindent{\bf #1: #2} \vspace{0.5em}
\hrule \vspace{.10in}}
\renewcommand{\part}[1] {\vspace{.10in} {\bf (#1)}}

\newcommand{\myname}{Alex Havrilla}
\newcommand{\myandrew}{alumhavr}
\newcommand{\myhwnum}{Hw 1}

\newtheorem{theorem}{Theorem}
\newtheorem{prop}{Prop}
\theoremstyle{remark}
\newtheorem{lemma}{Lemma}
\newtheorem{remark}{Remark}
\newtheorem{defi}{Def}
\newtheorem{apps}{Application}
\newtheorem{quest}{Question}
\newtheorem{ans}{Answer}
\newtheorem{interest}{Interesting}
\newtheorem{theme}{Theme}
\newtheorem{back}{Background}
\newtheorem{idea}{Idea}
\newtheorem{example}{Example}

\setlength{\parindent}{0pt}
\setlength{\parskip}{5pt plus 1pt}
 
\pagestyle{fancyplain}
\lhead{\fancyplain{}{\textbf{HW\myhwnum}}}      % Note the different brackets!
\rhead{\fancyplain{}{\myname\\ \myandrew}}
\chead{\fancyplain{}{\mycourse}}

\linespread{1.3}

\title{Log : 10/18 - Current}

\begin{document}

\maketitle

\section{10/18}

\subsection{Differential Geometry}

\begin{verbatim}
	https://en.wikipedia.org/wiki/Riemannian_connection_on_a_surface
\end{verbatim}

\section{12/21}

\subsection{Automated Theorem Proving}

\subsubsection{Open Logic}

\begin{verbatim}
	http://builds.openlogicproject.org/
\end{verbatim}

Covers set theory, modal logic, model theory, computation, intuitionist models


\subsubsection{Lean}

\begin{verbatim}
	https://github.com/leanprover-community/mathematics_in_lean
\end{verbatim}

Tutorial: 

\begin{verbatim}
	https://github.com/leanprover-community/mathematics_in_lean
\end{verbatim}

\begin{back}
	Formal language setting: Dependent type theory
\end{back}

\begin{quest}
	What does $\leftarrow$ do
\end{quest}
\begin{ans}
	Applies reverse rule(elimation rule). Like $\leftarrow \text{mul\_assoc } $  looks for a + (c + d) to turn into a + c +d instead of other way 
\end{ans}

\begin{remark}
	Arguments to tactics are curried
\end{remark}

\begin{theme}
	Making mathematics more empirical/feedback oriented! Especially brilliant because it mirrors coding process so effectively.
\end{theme}

\begin{remark}
	Can rewrite any statement, assumption or goal
\end{remark}

\begin{prop}
	apply tactic matches conclusion of theorem to goal and makes hypotheses new goal
\end{prop}

\begin{prop}
	exact tactic finishes proof with full apply(if given proof matches goal exactly)
\end{prop}

\begin{remark}
	For working backwards
\end{remark}

\begin{quest}
	Not really sure of difference between apply and exact
\end{quest}

\section{12/29}

\subsection{Deep RL}

\begin{verbatim}
	https://cmudeeprl.github.io/Spring202010403website/lectures/
\end{verbatim}

\begin{idea}
	Using reinforcement learning in automated proof theory.
\end{idea}

\begin{prop}
	In RL often cannot use gradient optimization, in contrast to supervised learning. So instead we use non-gradient optimization methods and gradient estimators
\end{prop}

\begin{prop}

	\begin{verbatim}

		"it is comparatively easy to make computers 
		exhibit adult level performance on intelligence tests or
		 playing checkers, and difficult or impossible to give them the skills
		  of a one-year-old when it comes to perception and mobility"
	\end{verbatim}
	Hans Moravec
\end{prop}

\section{12/31}

\subsection{Northwestern Reserach}

\textbf{Transport Model for Feature Extraction}

\begin{verbatim}
	https://arxiv.org/pdf/1910.14543.pdf
\end{verbatim}

\begin{remark}
	Well known techniques for (nonlinear) feature extraction:
	\begin{itemize}
		\item Kernel PCA
		\item isomap
		\item locally linear embeddings
		\item laplacian eigenmaps
	\end{itemize}
\end{remark}

\begin{defi}
	Transport operator
	\begin{align*}
		Ty = Ly - div(vy)
	\end{align*}
	for some anti-symmetric matrix v and laplacian L
\end{defi}

\textbf{On The Energy Landscape of Spherical Spin Glass : p-Spin}

\begin{verbatim}
	https://arxiv.org/pdf/1702.08906.pdf
\end{verbatim}



\subsection{Tensor Contraction}

\begin{verbatim}
	https://www.quora.com/What-is
	-Tensor-contraction-
	How-to-compute-tensor-contraction
\end{verbatim}

\section{1/10}

\subsection{Set}

\begin{remark}
	Raw practice has allowed me to find sets quickly subconciously. Incoporating an algorithm to smooth out bias would be helpful.

	For example I first always look for homochromatic sets. This induces a bias that I am then slow to fix. Explicitly looking for other pairs would help.

	Cultivate useful habits(mental intuitions) and prune others.

	Still important to identify characteristics with surplus

	Good illutstration of the power of shifting perspectives and the necessity/efficiency of raw practice/intuition

	Methods for identifying all diff: identify bottlenecks and work from those. Or raw search

	Concious overhead corrupts subconciuos pattern recoginition(efficiency).
\end{remark}

\subsection{Chess}

\begin{remark}
	Don't make silly mistakes
\end{remark}

\section{1/11}

\subsection{Variational Techniques in Stochastic Geometry}

\begin{remark}
	Looking at random variables on graphs
\end{remark}

\begin{defi}
	U-stat : something nice
\end{defi}

\begin{defi}
	$1_{0 \to_{Z({\alpha})}} \partial B_n(0)}$ is even we can raverse to boundary only on boolean occupied area
\end{defi}

\begin{remark}
	Goal is to control second order properties(variances). 
\end{remark}

\begin{defi}
	Mecke dormula encompes results for poisson process
\end{defi}

\begin{remark}
	OU semigroup: behaves differently on poisson vs. gaussian vs. hypercube measures. Is non hypercontractice ie. no log-sobolev
\end{remark}

\begin{remark}
	Charlize polynomials are orthogonal under poisson density
\end{remark}

\subsection{Chess}

\begin{remark}
	Want to compute faster somehow. Spend more time computign when it's not my turn. To compute efficiently think ADVERSARIALLY(what does my opponent want?)
\end{remark}

\begin{remark}
	Can't be tunnel visiond.
\end{remark}

\begin{remark}
	Don't mentally slack when ahead. Be ruthless
\end{remark}

\begin{remark}
	At least for now, while I'm developing intuition, mitigate unnecessary risks. Don't make moves that worsen my position
\end{remark}

\begin{remark}
	Don't worsen your position. Find tactics. Have a plan
\begin{remark}

\begin{remark}
	In response to he's been aggressive with sicilian and winning: Lately he's a little bit of a mirror. Showing your stupidity to opponents.
\end{remark}

\begin{remark}
	Tactics flow from superior position. Squeeze your opponent. Don't give opportunity for chances
\end{remark}

\begin{remark}
	Protect your king sufficietly. don't leave open to checks with tempo when attacking. Watch for poisoned pawns. Play for time when need be
\end{remark}

\begin{remark}
	Look for pawn fork tactics more. Higher level players seem to make much better use of pawns, as attackings
\end{remark}

\section{1/26}

\subsection{Chess}

\begin{remark}
	Need to sometimes be constrictor like, not greedy in endgame with pawns. In general endgame is very scary, want to work on. See
	\begin{verbatim}
		https://www.chess.com/a/QuDi3FgiXAX8
	\end{verbatim}
\end{remark}

\begin{remark}
	Don't lose the game in your desire to win.
	\begin{verbatim}
		https://www.chess.com/a/2YpuPr2bxXAX8
	\end{verbatim}
\end{remark}

\begin{remark}
	Endgame principles: Keep king closer to pawn mass than opponents. Get pawns as far forward as possible
\end{remark}

\begin{remark}
	Calculate things through. Most people really have no idea what they're doing and just go through hoping it works
\end{remark}

\begin{remark}
	When I don't feel like being profalactic play aggressively. When I do play wel
\end{remark}

\begin{remark}
	If I'm feeling lazy, simplify and try to play conservatively. Be somewhat aggressive but not comittally
\end{remark}

\section{1/28}

\subsection{Goals}

\begin{enumerate}
	\item 1400 chess
	\item 3 pages of thesis
	\item work on research
\end{enumerate}

\subsection{Chess}

\begin{remark}
	Getting a pawn down its file to pressure opposing king extremely powerful. 
	\begin{verbatim}
		https://www.youtube.com/watch?v=cevjjS9w0vM
	\end{verbatim}
	Giri converts to a miraculous win against dominant bishop
\end{remark}

\begin{remark}
	Principles of least effort chess(and in general least effort whatever). Key is to put in minimal effort/reps while still getting benefit/preventing burnout. Do as much as I can with as little exposure

	Tactic/improve chess. Keep improving while taking advantage of tactics when possible. Easier to not think about grand strategy. When possible incorporate strategy. Prevent positional corruption until conversion

\end{remark}

\begin{remark}
	When finding a tactic always look for the counterplay.
\end{remark}

\begin{remark}
	Example of punishing aggressive queen: Great tradeoff positionally for less material: 
	\begin{verbatim}
		https://www.chess.com/a/357WvmXNEXAX8
	\end{verbatim}
\end{remark}

\begin{remark}
	Losses are opporunities for learning/improvement. Review carefully and try again. Example: Note how I could have continued my kingside attack but didn't: 
	\begin{verbatim}
		https://www.chess.com/a/2qqkNdNvJXAX8
	\end{verbatim}
\end{remark}

\begin{remark}
	Devoretsky's endgame manual: reccomendation
\end{remark}

\begin{remark}
	If I can perform when I'm burned out then I should always be able to perform. Note: difference between burnout and imbalance(I do well when I'm feeling good. Key is to not let losing streak make me feel bad).  
\end{remark}

\section{1/30}

\subsection{Chess}

\begin{remark}
	Want instincts to align with best practice. Especially in tactics trainer. Reduces need for computation. Also improve computation speed.
\end{remark}

\begin{remark}
	Don't give value to opponents pieces useless pieces.
\end{remark}

\begin{example}
	Back and forth game between Giri and Firouzja Tata Steel 2021: 
	\begin{verbatim}
		https://www.youtube.com/watch?v=0H9QLP5giAA
	\end{verbatim}
\end{example}

\begin{remark}
	It seems I really need to warm into chess(or at least tactics trainer) to remind myself of proper mindset. Effect seems to wear off pretty quickly(like doing deep math). Also if unable to put in prerequisite thought because of distraction, should not be doing.
\end{remark}

\begin{remark}
	Visualization trick: don't look at board.
\end{remark}

\begin{remark}
	Important to learn enough opening theory to not fuck myself.
\end{remark}

\begin{remark}
	It's instructional to note how the board is set up when you enter a tactic. Threats and counterchanes tend to be minimized. 
\end{remark}

\begin{remark}
	Should also learn some endame theory.
\end{remark}

\begin{remark}
	At some point should try to codify decision making process in tactics, like in set.
\end{remark}

\begin{remark}
	I give up center control too easily.
\end{remark}

\section{1/31}

\subsection{Chess}

\begin{remark}
	Often times it seems the game is an art of looking for chances.
\end{Remark}

\begin{remark}
	A game of good tactics, positioning, and distraction: 
	\begin{verbatim}
		https://www.chess.com/analysis/game/live/6409188506
	\end{verbatim}
\end{remark}


\end{document}

