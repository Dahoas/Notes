\documentclass[11pt]{article}
%you can look for fun LaTeX packages to use hereasdf

\usepackage{amsmath}
\usepackage{amssymb}
\usepackage{fancyhdr}
\usepackage{amsthm}

\usepackage{graphicx}
\usepackage{dcolumn}
\usepackage{bm}

%fun commands for fun sets
%make sure to use these in math mode
\newcommand{\Z}{\mathbb{Z}}
\newcommand{\R}{\mathbb{R}}
\newcommand{\N}{\mathbb{N}}
\newcommand{\C}{\mathbb{C}}
\newcommand{\m}{\mathcal{M}}
\newcommand{\Tt}{\mathcal{T}}
\newcommand{\pa}{\partial}
\newcommand{\dD}{\mathcal{D}}
\newcommand{\E}{\mathbb{E}}



\oddsidemargin0cm
\topmargin-2cm    
\textwidth16.5cm   
\textheight23.5cm  

\newcommand{\question}[2] {\vspace{.25in} \hrule\vspace{0.5em}
\noindent{\bf #1: #2} \vspace{0.5em}
\hrule \vspace{.10in}}
\renewcommand{\part}[1] {\vspace{.10in} {\bf (#1)}}

\newcommand{\myname}{Alex Havrilla}
\newcommand{\myandrew}{alumhavr}
\newcommand{\myhwnum}{Hw 1}

\newtheorem{theorem}{Theorem}
\newtheorem{prop}{Prop}
\theoremstyle{remark}
\newtheorem{lemma}{Lemma}
\newtheorem{remark}{Remark}
\newtheorem{defi}{Def}
\newtheorem{apps}{Application}
\newtheorem{quest}{Question}
\newtheorem{ans}{Answer}
\newtheorem{interest}{Interesting}
\newtheorem{theme}{Theme}
\newtheorem{back}{Background}
\newtheorem{idea}{Idea}
\newtheorem{example}{Example}

\setlength{\parindent}{0pt}
\setlength{\parskip}{5pt plus 1pt}
 
\pagestyle{fancyplain}
\lhead{\fancyplain{}{\textbf{HW\myhwnum}}}      % Note the different brackets!
\rhead{\fancyplain{}{\myname\\ \myandrew}}
\chead{\fancyplain{}{\mycourse}}

\linespread{1.3}

\title{Technical Animation}

\begin{document}

\maketitle

\section{Introductions}

\textbf{TAG:} TechnicalAnimation

\begin{interest}
	TA Arjun is interested in PDEs and numerical simulation.
\end{interest}

\begin{remark}
	Course Website:
	\begin{verbatim}
		http://graphics.cs.cmu.edu/nsp/course/15464-s21/www/
	\end{verbatim}

	\textit{Computer Animation: Algorithms and Techniques} is the course textbook. In drive.
\end{remark}

\begin{quest}
	Does greater physical simulation accuarcy lead to a less palatable viewing experience? 
\end{quest}

\begin{ans}
	Not sure but often directors will personify animations and we have different parameters to give differenter personfications. For example "angry storm".
\end{ans}

\begin{ans}
	It seems exaggerated motion is often more digestestible(think actors for example). Often used actors in motion capture
\end{ans}

\begin{interest}
	Rig Net: automatically rigging meshes. Note: rigging is process of jointing meshes, providing structure/skeleton.
\end{interest}

\begin{remark}
	Beginning of rigging: find medial axis of geometry and impose some structure.
\end{remark}

\subsection{Examples in Practice}

\textbf{TAG:} TechnicalAnimation

\begin{remark}
	L-systems developed to describe plant structures and generation.
\end{remark}

\begin{remark}
	Tools for good animation: The Anmimators Survival Kit.
\end{remark}

\begin{remark}
	Idea behind rigging: for easy animating want ball control points you can manipulate for convenience.
\end{remark}

\begin{remark}
	Cloth simulation involves a mesh...
	Cloth intersection problems in Pixar's Coco:
	\begin{verbatim}
		https://www.researchgate.net/publication/326907399_Better_collisions_and_faster_cloth_for_Pixar's_Coco
	\end{verbatim}
\end{remark}

\begin{remark}
	Traditional animation: keyframing. 

	New variant: procedural animation. Often used for crowd animation.
\end{remark}

\begin{interest}
	Interesting site:
	\begin{verbatim}
	www.massivesoftware.com
	\end{verbatim}
\end{interest}

\begin{interest}
	Character controller using Motion VAEs interesting.
\end{interst}



\end{document}

