\documentclass[11pt]{article}
%you can look for fun LaTeX packages to use hereasdf

\usepackage{amsmath}
\usepackage{amssymb}
\usepackage{fancyhdr}
\usepackage{amsthm}

\usepackage{graphicx}
\usepackage{dcolumn}
\usepackage{bm}

%fun commands for fun sets
%make sure to use these in math mode
\newcommand{\Z}{\mathbb{Z}}
\newcommand{\R}{\mathbb{R}}
\newcommand{\N}{\mathbb{N}}
\newcommand{\C}{\mathbb{C}}
\newcommand{\m}{\mathcal{M}}
\newcommand{\Tt}{\mathcal{T}}
\newcommand{\pa}{\partial}
\newcommand{\dD}{\mathcal{D}}
\newcommand{\E}{\mathbb{E}}
\newcommand{\norm}[1]{\left\lvert\left\lvert#1\right\rvert\right\rvert}
\newcommand{\normOne}[1]{\left\lvert#1\right\rvert}
\newcommand{\lparen}{\left(}
\newcommand{\rparen}{\right)}



\oddsidemargin0cm
\topmargin-2cm    
\textwidth16.5cm   
\textheight23.5cm  

\newcommand{\question}[2] {\vspace{.25in} \hrule\vspace{0.5em}
\noindent{\bf #1: #2} \vspace{0.5em}
\hrule \vspace{.10in}}
\renewcommand{\part}[1] {\vspace{.10in} {\bf (#1)}}

\newcommand{\myname}{Alex Havrilla}
\newcommand{\myandrew}{alumhavr}
\newcommand{\myhwnum}{Hw 1}

\newtheorem{theorem}{Theorem}
\newtheorem{prop}{Prop}
\theoremstyle{remark}
\newtheorem{lemma}{Lemma}
\newtheorem{remark}{Remark}
\newtheorem{defi}{Def}
\newtheorem{apps}{Application}
\newtheorem{quest}{Question}
\newtheorem{ans}{Answer}
\newtheorem{interest}{Interesting}
\newtheorem{theme}{Theme}
\newtheorem{back}{Background}
\newtheorem{idea}{Idea}
\newtheorem{example}{Example}

\setlength{\parindent}{0pt}
\setlength{\parskip}{5pt plus 1pt}
 
\pagestyle{fancyplain}
\lhead{\fancyplain{}{\textbf{HW\myhwnum}}}      % Note the different brackets!
\rhead{\fancyplain{}{\myname\\ \myandrew}}
\chead{\fancyplain{}{\mycourse}}

\linespread{1.3}

\title{April Log}

\begin{document}

\maketitle

\section{4/1}

\subsection{Goals}

\begin{enumerate}
	\item Review Complex - Crush exam
\end{enumerate}

\subsection{To Do}

\begin{enumerate}
	\item Schedule thesis 
	\item Submit MCS honors form
	\item Commit to Georgia Tech - Done
\end{enumerate}


\subsection{Complex Review}

\begin{theorem}
	Let $\hat{f}$ be such that 
	\begin{align*}
		|\hat{f}(\xi)| \leq Ae^{-2\pi a |\xi|}
	\end{align*}
	for some constants $A,a > 0$. 

	Then f is the restriction to $\R$ of a function holomorphic in the strip $S_b = \{z \in \C : |Im(z)| < b\}$ for $0 < b < a$. 
\end{theorem}

\begin{proof}
	We have the limiting approximation
	\begin{align*}
		f_n(z) = \int_{-n}^n \hat{f}(\xi)e^{2\pi i \xi}d\xi
	\end{align*}

	Can be extended to a definition in the strip $S_b$ by our assumption on boundedness of $\hat{f}$. Entireness comes from fact we are integrating a holomorphic integrand which is a limit of holomorphic functions(proved via Morera).

	We have uniform convergence to f and thus f is holomorphic(as uniform limit of holomorphic functions).
\end{proof}

\begin{remark}
	Note this implies f and $\hat{f}$ cannot both have compact support unless $f=0$. Since if f vanishes in an open interval then since it's holomorphic it must be 0 everywhere.
\end{remark}

\begin{theorem}
	Let f be continuous with moderate decrease on $\R$. Then f has an extension to the complex plane that is entire with $|f(z)| \leq A e^{2\pi M |z|}$ for some $A > 0$ if and only if $\hat{f}$ is supported in the interval $[-M,M]$. 
\end{theorem}

\begin{proof}
	Suppose $\hat{f}$ is supported in $[-M,M]$. Then
	\begin{align*}
		f(x) = \int_{-M}^M\hat{f}(\xi)e^{2 \pi i x \xi}d\xi \leq ||\hat{f}||_{\infty} \int_{-M}^M |e^{2 \pi i x \xi}|d\xi \leq ||\hat{f}||_{\infty} 2M e^{2 \pi M |x|}
	\end{align*}

	where we bound $|e^{2\pi i x \xi}|$ with $e^{2 \pi M |x|}$. Further f is entire since it is an integral of holomorphic integrand.

	Now suppose $|f(x)| \leq A e^{2 \pi M |x|}$ and entire. Two observations:

	1. The above argument gives us the stronger bound of $|f(z)| \leq e^{2 \pi M |y|}$ instead of in terms of $|z|$

	2. But this is not enough as we need some control over decay at $\infty$

	Step 1. Use the stronger bound $|f(z)| \leq \frac{e^{2\pi M |y|}}{1+x^2}$ to show $\hat{f}(z) = 0$ when $|z| > M$

	Step 2. Use slightly stronger bound
	$|f(z)| \leq e^{2\pi M |y|}$. 
\end{proof}

\begin{remark}
	Note this means $\hat{f}$ cannot be extended when $f$ is exponentially bounded and entire since otherwise it is 0.
\end{remark}

\begin{theme}
	A key idea here is to shift the fourier transform by some imaginary offset $iy$ and note this should still agree with real case via contour argument, as sides go to $\infty$.
\end{theme}

List of important tools:

\begin{itemize}
	\item Phragmen-Lindelof: Allows us to uniformly bound holomorphic functions over unbounded sections assuming uniform boundedness on the bondary and exponential boundedness inside
\end{itemize}

Proof Chains

\begin{itemize}
	\item Maximum-modulus $\implies$ Phragmen Lindelof $\implies$ Paley-Wiener
\end{itemize}

\begin{remark}
	Often limiting arguments are examples of a reduction of more structure to less, like in the case of Phargmen Lindelof where we use $e^{-\epsilon\pi|z|^2}$ as a subduing function.

	Let the limiting structure be the reducer, and the target structure be the reduced. Then the form of the reducer is guided by the reduced. In this case we need to recover the idenitty in the limit, and we need some sufficient decay.
\end{remark}

Problems:

\begin{prop}
	\begin{align*}
		\sum_{n=-\infty}^{\infty} \frac{1}{(\tau + n)^k} = \frac{(2\pi i)^{k-1}}{(k-1)!} \sum_{m=1}^{\infty} m^{k-1}e^{2 \pi i m \tau}
	\end{align*}
\end{prop}

\begin{proof}
	First inclination was to take $f(x) = \frac{1}{(\tau+z)^k}$ but it works better if we take $f(z) = z^{k-1}e^{2\pi i \tau z}$

	\begin{verbatim}
		https://math.stackexchange.com/questions/2927148/the-fourier-transform-of-fx-xiy-k-for-y0-k2
	\end{verbatim}
\end{proof}

\begin{remark}
	Difficulties: Picking right function to take fourier transform of. Didn't know how to evaluate fourier transform.

	Tricks: Picking right function, evaluating FT using integration by parts
\end{remark}

\begin{prop}
	Show if $Im(\tau) > 0$
	\begin{align*}
		\frac{(2\pi i)}{1!} \sum_{m=1}^{\infty} me^{2 \pi i m \tau} = \frac{\pi^2}{sin(\pi \tau)^2}
	\end{align*}
\end{prop}

\begin{proof}
	Nothing too tricky. Must evalute the sum via exponential series which can be done since $e^{2 \pi i m \tau} \to 0$ as $m\to \infty$ since $Im(\tau) > 0$. If the sum is $f$ is the primitive $F$. We can evalute F then differentiate. Differentiation can be justified via DCT and the exponential convergence.
\end{proof}

\begin{remark}
	Difficulties: none

	Tricks: Using exponential power series closed formula and integrating to simplify sum.
\end{remark}

Goal: Do all SS problems in chapter 4

1.

\begin{proof}
	Key point is that $\int_{-\infty}^tf(x)dx = 0$ for every t hence must be 0. 
\end{proof}



\begin{quest}
	When can a functions fourier transform be extended to be entire?
\end{quest}

\begin{ans}
	Sufficiently when we have sufficient decrease to bound the integral and apply DCT(like with gaussian). 

	Necessary?
\end{ans}


\section{4/2}

\subsection{Goals}

\begin{enumerate}
	\item Finish DRL
	\item Start and do as much of evolution as possible
	\item Work on thesis
	\item Practice Jazz
	\item Clean list
\end{enumerate}

\subsection{To Do}

\begin{enumerate}
	\item Schedule thesis 
	\item Submit MCS honors form
\end{enumerate}

\subsection{Exam Review}

1. 
\begin{prop}
Fix $z_1,...,z_n \in \mathbb{S}^1$. Wanted to show we can find z on unit circle such that

\begin{align*}
	f(z) = \prod_{i=1}^n |z-z_i| = 1
\end{align*}

\end{prop}

\begin{proof}
	I tried an elementary method arguing the roots of unity are the worst case.

	Slicker to notice $f(0) = 1$(I did not notice this) and then realize it must attain its maximum on boundary which is strictly greater than 1. And done by continuity since $f(z_j) = 0$(I did realize this). 
\end{proof}

\begin{prop}
	If f entire with $|f(z)| \leq e^{-ax^2+by^2}$ we want to show
	\begin{align*}
		|\hat{f}(\xi)| \leq e^{-a'\xi^2} \quad \xi \in \R
	\end{align*}
\end{prop}

\begin{proof}
	To show this should have contour integrated along $x-iy$ off the real line but I didn't do this(thought of it right at end). Instead tried some weak fourier inversion stuff and that's it.
\end{proof}

\subsection{DRL}

OH:

\begin{remark}
	Q-learning is basically q-value iteration but over an infinite input space. So instead of updating everything in an order we do so randomly based on the actions we hav selected. This is why we have target and new network

	Normal update is done in passes, where pass k computes kth values and we update by taking max over reward plus discount plus max over next actions of next state.
\end{remark}

Tricks: use bitmasks for dones to avoid for loops

\section{4/3}

\subsection{Goals}


\begin{enumerate}
	\item Start and do as much of evolution as possible
	\item Work on thesis
	\item Practice Jazz
	\item Clean list
\end{enumerate}

\subsection{To Do}

\begin{enumerate}
	\item Schedule thesis 
	\item Look for internships
\end{enumerate}

\sectin{4/5}

\subsection{Goals}

\begin{enumerate}
	\item Grind thesis
	\item Finish DRL
	\item Complex homework
	\item Clean list
\end{enumerate}

\subsection{To Do}

\begin{enumerate}
	\item Submit MCS form
\end{enumerate}

\subsection{DRL}

Policy Gradient vs DQN:

One does gradient descent on an objective and the other an iterative value update.

\begin{verbatim}
	https://flyyufelix.github.io/2017/10/12/dqn-vs-pg.html
\end{verbatim}

Policy gradients seek to directly optimize over policy space(instead of learning q-values and construcing the policy via a max). The drawback of policy gradient is the high variance of estimating gradients. 

\begin{quest}
	Why do policy gradient methods have high variance?
\end{quest}

\begin{ans}
	Just because gradient can look wildy different depending on directions of trajectories.
\end{ans}

\begin{verbatim}
	https://www.quora.com/Why-does-the-policy-gradient-method-have-a-high-variance
\end{verbatim}

\subsection{Complex Analysis}

\section{4/6}

\subsection{Modeling Evolution}

Zhang 2020: Changes in contact patterns shape the dynamics of the COVID-19 outbreak in China 

\section{4/7}

\subsection{Goals}

\begin{enumerate}
	\item Finish complex
	\item Review DRL
	\item Grind thesis
\end{enumerate}

\section{4/10}

\subsection{Goals}

\begin{enumerate}
	\item Grind Thesis
	\item Start DRL
	\item Work on modeling evolution
	\item Start complex
\end{enumerate}

\subsection{Modeling Evolution Project}

See project notes

\subsection{Complex Analysis}

Use of the Scharz Lemma: establishes all functions with $g(0) = 0$ and automorphism of $\mathbb{D}$ as rotations in proof of characterization of automorphisms on $\mathbb{D}$

\section{4/11}

\subsection{Goals}

\begin{enumerate}
	\item Finish Thesis
\end{enumerate}

\section{4/13}

\subsection{Goals}

\begin{enumerate}
	\item Finish Thesis
	\item Finish complex
	\item Email people
	\item Start DRL
	\item Propel homework
\end{enumerate}

\subsection{To Do}

\begin{enumerate}
	\item Netbox interview thing
	\item Email tkocz and other masters people(gilchrist, ago, odonnel, gu)
\end{enumerate}

\subsection{Complex Analysis}

Green's theorem.

\begin{verbatim}
	https://tutorial.math.lamar.edu/classes/calciii/GreensTheorem.aspx
\end{verbatim}

For any area $A$ 
\begin{equation*}
	A = \int_C xdy = -\int_C ydx = \frac{1}{2}\int_C xdy - y dx
\end{equation*} if we suppose $Q_x - P_y = 1$. 

$f : U \to \mathbb{C}$ holomorphic is \textit{Univalent} or \textit{Schlicht} if it is injective. Univalent := holomorphic and injective

Discussion of Koebe 1/4 theorem and Gronwall area inequality.

\begin{verbatim}
	http://wwwf.imperial.ac.uk/~dcheragh/Teaching/2016-F-GCA/2016-F-GCA-Ch6.pdf
\end{verbatim}

\begin{equation*}
	\int_{f(A)}g(x)dx = \int_Ag(f(x))f'(x)dx+
\end{equation*}

\section{4/14}

\subsection{Goals}

\begin{enumerate}
	\item Finish Thesis
	\item Finish complex
	\item start drl
\end{enumerate}

\subsection{To Do}

\begin{enumerate}
	\item Netbox interview thing
	\item Email tkocz and other masters people(gilchrist, gu)
\end{enumerate}

\subsection{Relaxation}

\begin{enumerate}
	\item Set
	\item Reading fantasy
\end{enumerate}

\section{4/15}

\subsection{Goals}

\begin{enumerate}
	\item Finish Thesis
	\item start drl
	\item Fix propel
\end{enumerate}

\subsection{To Do}

\begin{enumerate}
	\item Netbox interview thing
	\item Email tkocz and other masters people(gilchrist, gu)
\end{enumerate}

\subsection{Relaxation}

\begin{enumerate}
	\item Set
	\item Maia
\end{enumerate}

\section{4/16}

\subsection{Goals}

\begin{enumerate}
	\item Finish Thesis
	\item start drl
	\item Fix propel
\end{enumerate}

\subsection{To Do}

\begin{enumerate}
	\item Netbox interview thing
	\item Email tkocz and other masters people(gilchrist, gu)
\end{enumerate}

\subsection{Relaxation}

\begin{enumerate}
	\item Set
	\item Maia
\end{enumerate}

\section{4/17}

\subsection{Goals}

\begin{enumerate}
	\item Finish Thesis
	\item start drl
	\item study complex
	\item do engage essays
\end{enumerate}

\subsection{To Do}

\begin{enumerate}
	\item Netbox interview thing
	\item Email tkocz and other masters people(gilchrist, gu)
\end{enumerate}

\subsection{Relaxation}

\begin{enumerate}
	\item Set
	\item Read model theory
\end{enumerate}

\section{4/18}

\subsection{Goals}

\begin{enumerate}
	\item Finish Thesis
	\item start drl
	\item Study complex
\end{enumerate}

\subsection{To Do}

\begin{enumerate}
	\item Netbox interview thing
	\item Email tkocz and other masters people(gilchrist, gu)
\end{enumerate}

\subsection{Relaxation}

\begin{enumerate}
	\item Set
	\item to be determined
\end{enumerate}

\subsection{Complex Analysis Review}

Simple connectedness is a necessary condition for Riemann mapping theorem since $F$ holomoprhic implies F continuous and continuous functions preserve simple connectedness.

\subsubsection{Riemann Mapping Theorem}

\begin{theorem}\label{thm:RM}
	Let $\Omega \subseteq \mathbb{C}$ be open and simply connected. Then there exists bijective and holomorphic $\phi : \Omega \to \mathbb{D}$ i.e. they are conformally equivalent. Further for arbitrary $z_0 \in \Omega$ we can find unique $\phi_{z_0}$ such that $\phi_{z_0}(z_0) = 0$ with $|\phi_{z_0}'(z_0)| > 0$. 
\end{theorem}

Ingredients:

\begin{lemma}[Montel's Theorem]
	$\mathcal{F}$ unif. bdd. family of holo functions on $\Omega$. Then
	\begin{enumerate}
		\item $\mathcal{F}$ is equicontinuous on every compact subset K
		\item $\mathcal{F}$ is normal
	\end{enumerate}
\end{lemma}

So in effect families of holo functions that are uniformly bounded are automatically equicontinuous! This is what is needed to extract function from $\mathcal{F} = \{f : \Omega \to \mathbb{C} holo, injective,f(z_0) = 0\}$ using regularity.

\begin{proof}
	Pick $z,w \in K \subseteq \Omega$ close. $\normOne{z-w} < r$. Then using cauchy formula
	\begin{equation*}
		f(z)-f(w) = \frac{1}{2\pi i}\int_{D_{2r}(w)}f(\xi)(\frac{1}{\xi - z}-\frac{1}{\xi-w})d\xi
	\end{equation*}
	So since $f$ uniformly bounded simply need to bound $\frac{1}{\xi - z}-\frac{1}{\xi-w}$. Furher
	\begin{equation*}
		\normOne{\frac{1}{\xi - z}-\frac{1}{\xi-w}} \leq \normOne{\frac{w-z}{(w-\xi)(z-\xi)}} \leq \frac{\normOne{w-z}}{2r^2}
	\end{equation*}
	which is enough to close our integral estimate.

	Cauchy formula + uniform bound and some estimates on difference of fractions allow us to get equicontinuity 

	Then AA effectively gives us our second statement. We prove this for completeness. We extract the sequence $g_m$ via a diagonalization argument. Choose $\{\omega_j\}$ dense in $\Omega$ and countable. Then we choose our functions to converge on this set for each point by taking subsequences of subsequences. Then we choose our sequence to be the diagonal of this matrix construction. It remains to argue uniform convergence. Note we pick using bolzanno weierstrass and boundedness. Continuity + convergence on countable dense set controls whole function. Uniform continuity comes from picking covering set via compactness of size $\delta$ for $\epsilon$ and getting convergence among each of the circles and picking over the largest one(maxing).

	Diagonal picking + total boundedness gives uniform convergence from equicontinuity
\end{proof}

Key new ingredient: Cauchy's integral formula.

\begin{lemma}
	Uniform limits of holomorphic injective functions are injective or constant.
\end{lemma}

\begin{proof}
	Suppose $g = \lim_n g_n$ not injective. Then $z\neq w, g(z) = g(w)$. Want to show constant.

	MY ATTEMPT: 

	Derivative could be 0, or arbitrary difference could be 0. 

	Set $d_n(\xi) = g_n(\xi) - g_n(z)$. So $d_n(w) \to 0$ uniformly. AFSOC not, then this is an isolated 0. 

	My END:

	$\frac{1}{2\pi i}\int_{\partial D_{\epsilon}(w)} \frac{d'(\xi)}{d(\xi)}d\xi = 1$ WHY???

	But $\int_{\partial D_{\epsilon}(w)} \frac{d_n'(\xi)}{d_n(\xi)}d\xi = 0$ because holomorphic. Uniform convergence should give convergence of integral but insufficient.
\end{proof}

Key new ingredient integration of holomorphic functions. 

\begin{proof}[Proof of \label{thm:RM}]
	\textbf{Uniqueness:} straightforward. If two such maps do exist, $F$ and $G$, for $z_0$ we let $H(z) = \frac{F(z)}{H(z)}$ and note $H(z_0) = 1$. Note then the singularity is removable and otherwise $H$ holomorphic and is bounded. Hence it is constant and 1. 

	ALTERNATIVELY note $H = F \circ G^{-1}$ is an automorphism of $\mathbb{D}$. We know $H' = F'(G^{-1})\circ G^{-1}'$. Also $H(0) = 0$ implying its a scaling i.e. $H(z) = cz$ for some $c=H'(0)=1$. 

	So uniqueness follows by considering the compositition which must be an automorphism of $\mathbb{D}$ fixing $0$. Then since it must be a scaling we simply compute the derivative to see the scaling is actually identity. 

	\textbf{Existence:} $\mathcal{F} = \{f : \Omega \to \mathbb{C} holo, injective,f(z_0) = 0\}$ is compact. 

	Step 1: Reduce to $\Omega$ to $0 \in \Omega'\subseteq \mathbb{D}$. 

	Step 2: Extract $f$ with maximal $\normOne{f'(0)}$ and via uniform convergence show in $\mathcal{F}$. 

	Step 3: Show this $f$ onto. We do this by tweaking f so that $f'(0) > s$ the supremal $f'(0)$ if there exists $\alpha$ it does not hit in $\mathbb{D}$. 
\end{proof}

Note: square root defined via log.

\subsection{Chapter 8 Problems}

Resources:

\begin{verbatim}
	https://math.berkeley.edu/~murphy/185-Solutions7.pdf
\end{verbatim}

Problem 1: Crux is using Rouche to argue for sufficiently close $w$ that we at least two zeroes via lowest degree term of F power series being order two poly or more.

\section{4/19}

\subsection{Goals}

1. Work DRL
2. Do Propel
3. Do engage

\subsection{Relaxing}

1. Read PDE and data
2. Fantasy

\subsection{Jazz}

For 16 bar solo of same chord count 4 bars at a time

Good way to hold form together is to draw on melody(particularly when needing help).

Idea: abcd# descending and ascending chord to a and back

Listen to kind of blue album

\section{4/20}

\subsection{Goals}

1. DRL

2. Engage essays

3. Complex

\subsection{Relaxation}

1. PDEandML

2. Set

\subsection{DRL}

Advantage of simulated dynamics model is that it is much less expensive/costly to simulate runs in head than in environment.

Optimal control:

\begin{verbatim}
	https://ai.stackexchange.com/questions/11375/what-is-the-difference-between-reinforcement-learning-and-optimal-control/23908
\end{verbatim}

\section{4/21}

\subsection{Goals}

1. Grind DRL

2. Work complex

3. Clear head

\subsection{To Do}

1. Email tkocz - done

2. Do netbox thing

\subsection{Relaxation}

1. Set 

2. Read PDE and Data

\subsection{DRL Homework}

\section{4/22}

\subsection{Goals}

1. Grind DRL

2. Work complex

3. Model evolution

\subsection{Tkocz notes}

1. Justify expectatatio integral formula in Haagerup.

2. Make sure to get all full-stops

\subsection{DRL Notes}

Idea behind MPC is to predict $T$ steps in advance using optimizer. Then at the next step we iteratively update plan after stepping through environment. Plan formation and adaptation.

\section{4/24}

\subsection{DRL}

Homework 4:

	Questions:

	1. How is MPC used in Random policy selection

\section{4/25}

\subsection{Goals}

1. Finish DRL

2. Make reserach presentation

3. Modeling evolution

\subsection{To Do}

1. Netbox stuff

2. Email Gilchrist

3. Email to set up thesis 

\subsection{Relaxation}

1. Maia

2. QFT stuff

3. Read war and peace

\subsection{DRL}

Apparently eval shorthand for sess.run as default

\section{4/26}

\subsection{Goals}

1. Finish DRL

2. Make reserach presentation

3. Modeling evolution

4. Jazz

\subsection{To Do}

1. Netbox stuff

2. Email Gilchrist

3. Email to set up thesis 

4. Email exam

\subsection{Relaxation}

1. Maia

2. QFT stuff

3. Read Perdido street station

\subsection{DRL}

How to print in tensorflow computation? For example how to print losses?

Use pdb to debug tensorflow values

\subsection{Jazz}

Want to use roots as much as possible. Use voice leading between top note and melody note. 

Should play more open, for example move C to right hand. Thirds and sevenths are in right hand, not left hand. Sometimes sevenths can be doubled. This is particuarly true for solo piano playing.

Samba is heavy Brazilian rhythm, faster version of slow Tangerine. Most well written songs can be written in a variety of rhythmic settings. Good way to practice is to mess these up. 

Sometimes sharped nine vs flat nine can be interchanged via good voicing. recall good voicing means minimal movement

\section{4/27}

\subsection{Goals}

1. need to write scientific analysis for propel

2. Finish engage

3. Finish drl

4. Make research presentation

5. Work on evolution

\subsection{To Do}

1. Netbox

2. Email people about thesis

\subsection{Relaxation}

1. Set

2. Maia

\section{4/28}

\subsection{Goals}

1. need to write scientific analysis for propel

2. Finish engage

3. Finish drl

4. Make research presentation

5. Work on evolution

\subsection{To Do}

1. Netbox

2. Email people about thesis

\subsection{Relaxation}

1. Set

2. Read perdido station

\subsection{Thesis Presentation}

Basic Research:

https://blogs.ams.org/mathmentoringnetwork/2014/08/04/math-talk-preparing-your-conference-presentation/

https://control.com/technical-articles/what-is-model-predictive-control-mpc/

https://www.quora.com/What-are-the-pros-and-cons-of-model-based-and-model-free-reinforcement-learning

\subsection{Thesis Presentation Tex Bits}

https://www.codecogs.com/latex/eqneditor.php

https://math.asu.edu/resources/computer-resources/how-make-slides-latex

\[
	(\mathbb{E} |\sum_{i=1}^n a_i \epsilon_i|^p)^{1/p} \leq C_{p,q}(\mathbb{E} |\sum_{i=1}^n a_i \epsilon_i|^2)^{1/2}
\]

Thesis Remarks:

In general consider taking out some text.

Slide 2: Trivially true with $C_{p,q} = 1$ when $p < q$ via monotonicity of moments. The khintchine inequalities show equivalence of moments. Emphasize $C_{p,q}$ denotes the best possible constant.

He liked that I say khinthcine proved they exist. Example where these do not exist? When the moment is not finite kinda answers this.

Slide 3: Write down $C_{p,q}$ is supremum over $a,n$ for ratio of norms. Frame as optimization problem. Just say type and cotype are invariants of banach spaces, and averaging over random signs is a way to construt them. INclude stuff on Grothendieck and other thing as bullet list

Khintchine proved these to prove law of iteratd logarithm. This was his motivation(in 1923). Rediscovered by Hardy and Littlewood. Hardy first proved upper bound by $\sqrt{nlog(n)}$ refined by Khintchine(he knew about Hardy) by $\sqrt{nlog(log(n))}$. This tighter version later showed by Hardy(who did not know about Khintchine???).

Also projection not intersection

Slide 4: Instead of citing numbers puts names and dates

Slide 5: Just define usg for random variables, not random vectors

Slide 6: Have display somewhere saying constants are sharp because of central limit theorem. Say expectations converge via expecation argument with CLT

Slide 8: Say moment generating function. Tkocz would let go L' class. Write why random sings are type L

Slide 9: Bullet these. Or write as remarks/observations. Good to structure thigs.

Slide 11: Get rid of type L'. Run argument in finite case and then just pass to limit. Write out explicitly what the elementary symmetric functions are. Explicitly say we are comparing terms. Just most importantly define elementary symmetric functions

Slide 13: Write proof of lemma. Just say atom from product is a characteristic function. Just by taking IFT.

Slide 14: Make point this is khintchine inequality for $C_{p,2}$ since this is the variance. Fix variance. Try to minimizie and maximize the pth moment. Looking for extremals under this constraint. Variance is the second moment

Slide 16: Move before slide 15 and write example of sequence majorization. Fix typo. Also to mention proof key is to modify two coefficients at a time and use independence and show nonnegativity in a difference this way.

Slide 19: Bullet

Slide 20: Instead of writing out explicitly just write in terms of Gaussian

Slide 20: Proof relies on Gaussian swapping argument and using independence

Slide 21: Proof relies on convexity argument

Slide 22: Haagerup argument. Throughout formula for $c_1$. Equality in $n=1$ case. Would be nice to display fourier analytic integral representation

Slide 23: No need to display references

Note: No need to write our names on new theorems. Slides 10 and slide 14 

\subsection{DRL HW 4 MPC}

How is it possible random actions with ground truth and no MPC do better than MPC with a simulated dynamic? Does random even use MPC? Is CEM with learned dynamic MPC bad because learning bad? Maybe because the policy is MPC and random moves are done to generate episodes(instead of CEM fitness)?

\subsection{Modeling Evolution Notes}

\section{5/1}

\subsection{Goals}

\begin{itemize}
	\item Modeling evolution grind
	\item Finish presentation
\end{itemize}

\section{5/2}

\subsection{Thesis Notes}

Slide Point Targets
\begin{enumerate}
	\item Say what I'm presenting
	\item Go ove rsummary and make clear these are new results
	\item Mention triviality when $p < q$ and via monotonicity of moments. State existence which shows all moments equivalent.
	\item Averaging over random signs way to construct type and cotype
	\item Makes sure to mention most comparisons only known to second norm. 
	\item NA
	\item Mention used in polynomials to collect terms
	\item NA
	\item Make sure to emphasize sharp constants due to CLT
	\item NA
	\item Walk through why random signs type L
	\item Mention useful to then show Khintchine
	\item Emphasize Hadamard factorization
	\item Empahsize sharpness of constants via CLT
	\item NA
	\item NA
	\item NA
	\item Empahsize still khintchine because variance fixed and optimizing pth moment.
	\item Emphasize mass intuition(spreading v.s. congregating). Gaussian is spread out case.
	\item NA
	\item Connect back to mass metaphor
	\item NA
	\item NA
	\item NA
	\item NA
	\item NA
\end{enumerate}

should say second moment easy to compute

remove rotation invariance

Remove reference slide 7

remove type L'

a_j

Fix slide 21 grammar.

Should say I'm going to present new results on type L random variables and Symmetric discrete uniform random variables. I say this in the outline

Highlight random signs definition

Use box for definitions instead of highlighting

Metaphor for spreading and congregating mass useful

Sydney wants more pictures

\end{document}

