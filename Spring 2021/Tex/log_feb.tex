\documentclass[11pt]{article}
%you can look for fun LaTeX packages to use hereasdf

\usepackage{amsmath}
\usepackage{amssymb}
\usepackage{fancyhdr}
\usepackage{amsthm}

\usepackage{graphicx}
\usepackage{dcolumn}
\usepackage{bm}

%fun commands for fun sets
%make sure to use these in math mode
\newcommand{\Z}{\mathbb{Z}}
\newcommand{\R}{\mathbb{R}}
\newcommand{\N}{\mathbb{N}}
\newcommand{\C}{\mathbb{C}}
\newcommand{\m}{\mathcal{M}}
\newcommand{\Tt}{\mathcal{T}}
\newcommand{\pa}{\partial}
\newcommand{\dD}{\mathcal{D}}
\newcommand{\E}{\mathbb{E}}



\oddsidemargin0cm
\topmargin-2cm    
\textwidth16.5cm   
\textheight23.5cm  

\newcommand{\question}[2] {\vspace{.25in} \hrule\vspace{0.5em}
\noindent{\bf #1: #2} \vspace{0.5em}
\hrule \vspace{.10in}}
\renewcommand{\part}[1] {\vspace{.10in} {\bf (#1)}}

\newcommand{\myname}{Alex Havrilla}
\newcommand{\myandrew}{alumhavr}
\newcommand{\myhwnum}{Hw 1}

\newtheorem{theorem}{Theorem}
\newtheorem{prop}{Prop}
\theoremstyle{remark}
\newtheorem{lemma}{Lemma}
\newtheorem{remark}{Remark}
\newtheorem{defi}{Def}
\newtheorem{apps}{Application}
\newtheorem{quest}{Question}
\newtheorem{ans}{Answer}
\newtheorem{interest}{Interesting}
\newtheorem{theme}{Theme}
\newtheorem{back}{Background}
\newtheorem{idea}{Idea}
\newtheorem{example}{Example}

\setlength{\parindent}{0pt}
\setlength{\parskip}{5pt plus 1pt}
 
\pagestyle{fancyplain}
\lhead{\fancyplain{}{\textbf{HW\myhwnum}}}      % Note the different brackets!
\rhead{\fancyplain{}{\myname\\ \myandrew}}
\chead{\fancyplain{}{\mycourse}}

\linespread{1.3}

\title{Feb Log}

\begin{document}

\maketitle

\section{1/31}

\subsection{Chess}

\begin{remark}
	A blunder free game with weak positional moves:
	\begin{verbatim}
		https://www.chess.com/analysis/game/live/6409740211?tab=analysis
	\end{verbatim}
\end{remark}

\begin{remark}
	A complicated blunder filled game:
	\begin{verbatim}
		https://www.chess.com/a/CbAJ8Wm4XAX8
	\end{verbatim}	

	To Analyze:
\end{remark}

\subsection{Complex Analysis}

\section{2/1}

\subsection{Chess}

\begin{remark}
	Talk about a clean game:
	\begin{verbatim}
		https://www.chess.com/a/Gzp6PJxWXAX8
	\end{verbatim}
\end{remark}

\begin{remark}
	My first brilliant move!:
	\begin{verbatim}
		https://www.chess.com/a/2BfrDrz2JXAX8
	\end{verbatim}
\end{remark}



\subsection{Technical Animation}

\begin{interest}
	TA Arjun is interested in PDEs and numerical simulation.
\end{interest}

\begin{remark}
	Course Website:
	\begin{verbatim}
		http://graphics.cs.cmu.edu/nsp/course/15464-s21/www/
	\end{verbatim}

	\textit{Computer Animation: Algorithms and Techniques} is the course textbook. In drive.
\end{remark}

\begin{quest}
	Does greater physical simulation accuarcy lead to a less palatable viewing experience? 
\end{quest}

\begin{ans}
	Not sure but often directors will personify animations and we have different parameters to give differenter personfications. For example "angry storm".
\end{ans}

\begin{ans}
	It seems exaggerated motion is often more digestestible(think actors for example). Often used actors in motion capture
\end{ans}

\begin{interest}
	Rig Net: automatically rigging meshes. Note: rigging is process of jointing meshes, providing structure/skeleton.
\end{interest}

\begin{remark}
	Beginning of rigging: find medial axis of geometry and impose some structure.
\end{remark}

\subsection{On Lp Brunn-Minkowski Type Inequalities}

\textbf{Tag}: BrunnMinkowski

\begin{remark}
	V is $1/n$ concave measure w.r.t Minkowsi sum. Need normalizing $1/n$ powers 
\end{remark}

\begin{prop}
	\begin{align*}
		h_{K+L}(u) = h_K(u) + h_L(u)
	\end{align*}
\end{prop}

\begin{remark}
	\textbf{Brascamp-Lieb}

	$\alpha \geq -1/n, t\in [0,1]$. With $f,g,h : \mathbb{R}^n \to \mathbb{R}_+$ satisfy

$h((1-t)x + ty) \geq [(1-t)f(x)^{\alpha} + tg(y)^{\alpha}]^{1/\alpha}$ then

$\int_{R^n} h(x)dx \geq [(1-t)(\int_{R^n}f(x)dx)^{\alpha/1+n\alpha}+t(\int_{R^n}g(x)dx)^{\alpha/1+n\alpha}]^{1+n\alpha/\alpha}$

Prekopa lindler is $\alpha = 0$
\end{remark}

\begin{prop}
	\begin{align*}
		(1-t)X_s1_A \oplus_s t X_s 1_B = 1_{(1-t)A + tB}
	\end{align*}
\end{prop}

\begin{remark}
	Changing operator: minkwoski sum, to $l_p$ variants. 
\end{remark}

\begin{remark}
	Also some kind of interplay beteween functional inequalities and volume inequalities. Between supremal convolutions and Lp minkwoski sums.
\end{remark}

\subsection{PDEs and Data Analysis}

\textbf{TAG:} OptimalTransport

\begin{theme}
	The more assumptions you make on a measure the better approximation you can achieve
\end{theme}

\begin{interest}
	Shimaa is interested in stochastic BDEs. Wes interested in foundations of machine learning.
\end{interest}

\begin{theme}
	Look at a measure as some kind of energy landscape and the transport map as the process of rearranging mass.
\end{theme}

\begin{remark}
	Often transportation cost is $|x-y|^p$. 
\end{remark}

\begin{remark}
	Optimal transport minimizes transportation cost.
\end{remark}

\begin{theme}
	Goal is to find weaker problem which provides good solution to wider class of subproblems.
\end{theme}


\section{2/2}

\subsection{Goals}

\begin{enumerate}
	\item Chess: 1300 in blitz
	\item Research: 3 hours worked, some progress, email tkocz
	\item Thesis: 5 pages
	\item Homework: Animation
	\item Get glenn to agree to a time
\end{enumerate}

\subsection{DRL}

\textbf{TAG:} DRL

\begin{quest}
	what is computational design?
\end{quest}

\begin{remark}
	Course link: 
	\begin{verbatim}
		https://cmudeeprl.github.io/403_website/
	\end{verbatim}
\end{remark}

\begin{remark}
	Katerina F.
	\begin{verbatim}
		"My genes have strong priors from the world"
	\end{verbatim}
\end{remark}

\begin{remark}
	Inconsistent rewards lead to addiction.
\end{remark}

\begin{remark}
	For a long time large emphasis on discovering new behaiors in DRL. Now thinking we need to develop behavior repetoire and associate with some stimuli.
\end{remark}

\begin{remark}
	Curiosity, a desire to see new things, very intrinsically powerful.
\end{remark}

\begin{remark}
	Conor Igoe: 
	\begin{verbatim}
		For a fixed known opponent, the evolution of chess is markovian from the perspective of the main player.
	\end{verbatim}

	In some cases(such as driving) we need multiple frames/time steps to even attempt to play. But this can also be redefined as markovian by letting states correspond to multiple time steps.
\end{remark}

\begin{remark}
	Model vs. non-model based. Can we learn via simulation or not.
\end{remark}

\begin{remark}
	Cannot use gradient based optimization often in DRL. We can if we have a model.
\end{remark}

\subsection{The Embodiment Hypothesis}

\begin{remark}
	Link:
	\begin{verbatim}
		https://cogdev.sitehost.iu.edu/labwork/6_lessons.pdf
	\end{verbatim}
\end{remark}




\subsection{Modeling Evolution}

\end{document}

