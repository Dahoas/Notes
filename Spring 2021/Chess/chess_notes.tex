\documentclass[11pt]{article}
%you can look for fun LaTeX packages to use hereasdf

\usepackage{amsmath}
\usepackage{amssymb}
\usepackage{fancyhdr}
\usepackage{amsthm}

\usepackage{graphicx}
\usepackage{dcolumn}
\usepackage{bm}

%fun commands for fun sets
%make sure to use these in math mode
\newcommand{\Z}{\mathbb{Z}}
\newcommand{\R}{\mathbb{R}}
\newcommand{\N}{\mathbb{N}}
\newcommand{\C}{\mathbb{C}}
\newcommand{\m}{\mathcal{M}}
\newcommand{\Tt}{\mathcal{T}}
\newcommand{\pa}{\partial}
\newcommand{\dD}{\mathcal{D}}
\newcommand{\E}{\mathbb{E}}



\oddsidemargin0cm
\topmargin-2cm    
\textwidth16.5cm   
\textheight23.5cm  

\newcommand{\question}[2] {\vspace{.25in} \hrule\vspace{0.5em}
\noindent{\bf #1: #2} \vspace{0.5em}
\hrule \vspace{.10in}}
\renewcommand{\part}[1] {\vspace{.10in} {\bf (#1)}}

\newcommand{\myname}{Alex Havrilla}
\newcommand{\myandrew}{alumhavr}
\newcommand{\myhwnum}{Hw 1}

\newtheorem{theorem}{Theorem}
\newtheorem{prop}{Prop}
\theoremstyle{remark}
\newtheorem{lemma}{Lemma}
\newtheorem{remark}{Remark}
\newtheorem{defi}{Def}
\newtheorem{apps}{Application}
\newtheorem{quest}{Question}
\newtheorem{ans}{Answer}
\newtheorem{interest}{Interesting}
\newtheorem{theme}{Theme}
\newtheorem{back}{Background}
\newtheorem{idea}{Idea}
\newtheorem{example}{Example}

\setlength{\parindent}{0pt}
\setlength{\parskip}{5pt plus 1pt}
 
\pagestyle{fancyplain}
\lhead{\fancyplain{}{\textbf{HW\myhwnum}}}      % Note the different brackets!
\rhead{\fancyplain{}{\myname\\ \myandrew}}
\chead{\fancyplain{}{\mycourse}}

\linespread{1.3}

\title{Chess}

\begin{document}

\maketitle

\section{Openings}

\begin{remark}
	In response to he's been aggressive with sicilian and winning: Lately he's a little bit of a mirror. Showing your stupidity to opponents.
\end{remark}

\section{Middle Games}

\begin{remark}
	Tactics flow from superior position. Squeeze your opponent. Don't give opportunity for chances
\end{remark}

\section{Endgames}

\begin{remark}
	Need to sometimes be constrictor like, not greedy in endgame with pawns. In general endgame is very scary, want to work on. See
	\begin{verbatim}
		https://www.chess.com/a/QuDi3FgiXAX8
	\end{verbatim}
\end{remark}

\begin{remark}
	Endgame principles: Keep king closer to pawn mass than opponents. Get pawns as far forward as possible
\end{remark}

\section{Tactics}

\begin{remark}
	Getting a pawn down its file to pressure opposing king extremely powerful. 
	\begin{verbatim}
		https://www.youtube.com/watch?v=cevjjS9w0vM
	\end{verbatim}
	Giri converts to a miraculous win against dominant bishop
\end{remark}

\begin{remark}
	When finding a tactic always look for the counterplay.
\end{remark}

\begin{remark}
	Want instincts to align with best practice. Especially in tactics trainer. Reduces need for computation. Also improve computation speed.
\end{remark}

\begin{remark}
	A game of good tactics, positioning, and distraction: 
	\begin{verbatim}
		https://www.chess.com/analysis/game/live/6409188506
	\end{verbatim}
\end{remark}

\section{Strategy}

\begin{remark}
	Getting a pawn down its file to pressure opposing king extremely powerful. 
	\begin{verbatim}
		https://www.youtube.com/watch?v=cevjjS9w0vM
	\end{verbatim}
	Giri converts to a miraculous win against dominant bishop
\end{remark}

\begin{remark}
	Don't give value to opponents pieces useless pieces.
\end{remark}

\begin{remark}
	I give up center control too easily.
\end{remark}

\section{Good Games}

\begin{remark}
	Need to sometimes be constrictor like, not greedy in endgame with pawns. In general endgame is very scary, want to work on. See
	\begin{verbatim}
		https://www.chess.com/a/QuDi3FgiXAX8
	\end{verbatim}
\end{remark}

\begin{remark}
	Don't lose the game in your desire to win.
	\begin{verbatim}
		https://www.chess.com/a/2YpuPr2bxXAX8
	\end{verbatim}
\end{remark}

\begin{remark}
	Example of punishing aggressive queen: Great tradeoff positionally for less material: 
	\begin{verbatim}
		https://www.chess.com/a/357WvmXNEXAX8
	\end{verbatim}
\end{remark}

\begin{example}
	Back and forth game between Giri and Firouzja Tata Steel 2021: 
	\begin{verbatim}
		https://www.youtube.com/watch?v=0H9QLP5giAA
	\end{verbatim}
\end{example}

\begin{remark}
	A game of good tactics, positioning, and distraction: 
	\begin{verbatim}
		https://www.chess.com/analysis/game/live/6409188506
	\end{verbatim}
\end{remark}

\begin{remark}
	Talk about a clean game:
	\begin{verbatim}
		https://www.chess.com/a/Gzp6PJxWXAX8
	\end{verbatim}
\end{remark}

\begin{remark}
	My first brilliant move!:
	\begin{verbatim}
		https://www.chess.com/a/2BfrDrz2JXAX8
	\end{verbatim}
\end{remark}

\begin{remark}
	A blunder free game with weak positional moves:
	\begin{verbatim}
		https://www.chess.com/analysis/game/live/6409740211?tab=analysis
	\end{verbatim}
\end{remark}

\begin{remark}
	A complicated blunder filled game:
	\begin{verbatim}
		https://www.chess.com/a/CbAJ8Wm4XAX8
	\end{verbatim}	

	To Analyze:
\end{remark}

\section{Position vs. Aggression}

\begin{remark}
	When I don't feel like being profalactic play aggressively. When I do play wel
\end{remark}

\begin{remark}
	If I'm feeling lazy, simplify and try to play conservatively. Be somewhat aggressive but not comittally
\end{remark}

\begin{remark}
	Example of punishing aggressive queen: Great tradeoff positionally for less material: 
	\begin{verbatim}
		https://www.chess.com/a/357WvmXNEXAX8
	\end{verbatim}
\end{remark}

\section{Book Reccomendations}

\begin{remark}
	Devoretsky's endgame manual: reccomendation
\end{remark}

\section{Visualizing}

\begin{remark}
	Visualization trick: don't look at board.
\end{remark}

\section{Uncategorized}

\begin{remark}
	Don't make silly mistakes
\end{remark}

\begin{remark}
	Want to compute faster somehow. Spend more time computign when it's not my turn. To compute efficiently think ADVERSARIALLY(what does my opponent want?)
\end{remark}

\begin{remark}
	Can't be tunnel visiond.
\end{remark}

\begin{remark}
	Don't mentally slack when ahead. Be ruthless
\end{remark}

\begin{remark}
	At least for now, while I'm developing intuition, mitigate unnecessary risks. Don't make moves that worsen my position
\end{remark}

\begin{remark}
	Don't worsen your position. Find tactics. Have a plan
\begin{remark}

\begin{remark}
	Protect your king sufficietly. don't leave open to checks with tempo when attacking. Watch for poisoned pawns. Play for time when need be
\end{remark}

\begin{remark}
	Calculate things through. Most people really have no idea what they're doing and just go through hoping it works
\end{remark}

\begin{remark}
	Principles of least effort chess(and in general least effort whatever). Key is to put in minimal effort/reps while still getting benefit/preventing burnout. Do as much as I can with as little exposure

	Tactic/improve chess. Keep improving while taking advantage of tactics when possible. Easier to not think about grand strategy. When possible incorporate strategy. Prevent positional corruption until conversion

\end{remark}

\begin{remark}
	Losses are opporunities for learning/improvement. Review carefully and try again. Example: Note how I could have continued my kingside attack but didn't: 
	\begin{verbatim}
		https://www.chess.com/a/2qqkNdNvJXAX8
	\end{verbatim}
\end{remark}

\begin{remark}
	If I can perform when I'm burned out then I should always be able to perform. Note: difference between burnout and imbalance(I do well when I'm feeling good. Key is to not let losing streak make me feel bad).  
\end{remark}

\begin{remark}
	It seems I really need to warm into chess(or at least tactics trainer) to remind myself of proper mindset. Effect seems to wear off pretty quickly(like doing deep math). Also if unable to put in prerequisite thought because of distraction, should not be doing.
\end{remark}

\begin{remark}
	At some point should try to codify decision making process in tactics, like in set.
\end{remark}

\begin{remark}
	Often times it seems the game is an art of looking for chances.
\end{Remark}

\end{document}

