\documentclass[11pt]{article}
%you can look for fun LaTeX packages to use hereasdf

\usepackage{amsmath}
\usepackage{amssymb}
\usepackage{fancyhdr}
\usepackage{amsthm}

\usepackage{graphicx}
\usepackage{dcolumn}
\usepackage{bm}

%fun commands for fun sets
%make sure to use these in math mode
\newcommand{\Z}{\mathbb{Z}}
\newcommand{\R}{\mathbb{R}}
\newcommand{\N}{\mathbb{N}}
\newcommand{\C}{\mathbb{C}}
\newcommand{\m}{\mathcal{M}}
\newcommand{\Tt}{\mathcal{T}}
\newcommand{\pa}{\partial}
\newcommand{\dD}{\mathcal{D}}
\newcommand{\E}{\mathbb{E}}



\oddsidemargin0cm
\topmargin-2cm    
\textwidth16.5cm   
\textheight23.5cm  

\newcommand{\question}[2] {\vspace{.25in} \hrule\vspace{0.5em}
\noindent{\bf #1: #2} \vspace{0.5em}
\hrule \vspace{.10in}}
\renewcommand{\part}[1] {\vspace{.10in} {\bf (#1)}}

\newcommand{\myname}{Alex Havrilla}
\newcommand{\myandrew}{alumhavr}
\newcommand{\myhwnum}{Hw 1}

\newtheorem{theorem}{Theorem}
\newtheorem{prop}{Prop}
\theoremstyle{remark}
\newtheorem{lemma}{Lemma}
\newtheorem{remark}{Remark}
\newtheorem{defi}{Def}
\newtheorem{apps}{Application}
\newtheorem{quest}{Question}
\newtheorem{ans}{Answer}
\newtheorem{interest}{Interesting}
\newtheorem{theme}{Theme}
\newtheorem{back}{Background}
\newtheorem{idea}{Idea}
\newtheorem{example}{Example}

\setlength{\parindent}{0pt}
\setlength{\parskip}{5pt plus 1pt}
 
\pagestyle{fancyplain}
\lhead{\fancyplain{}{\textbf{HW\myhwnum}}}      % Note the different brackets!
\rhead{\fancyplain{}{\myname\\ \myandrew}}
\chead{\fancyplain{}{\mycourse}}

\linespread{1.3}

\title{Complex Analysis}

\begin{document}

\maketitle

\section{An Introduction}

\begin{idea}
	Proving non-analyticity. Show different limit in two diff. directions. Often useful to consider $h \in \R$ and $h \in i \R$. 
\end{idea}

\begin{prop}
	Often convenient to regard holomorphic definition as
	\begin{align*}
		f(z_0 + h) = f(z_0) + ah + o(h)
	\end{align*}

	for some $a$. Ie. the function is locally complex linear.
\end{prop}

\begin{remark}
	The above propsition is useful for proving chain rule.
\end{remark}

\subsection{Cauchy Riemann Equations}

\textbf{TAG:} ComplexAnalysis

\begin{remark}
	Cauchy riemann equations derived via simply differentiating f as a function of two varibles in real and comlex directions.
\end{remark}

\begin{quest}
	Cauchy riemann conditions are necessary. Are they sufficient?
\end{quest}

\begin{ans}
	Yes. 
	\begin{theorem}
		Let $f : \Omega \to \C$ with $f = u + iv$ and satisfying cauchy riemann. Then f is holomorphic.
	\end{theorem}
	\begin{proof}
		Fix $z_0 = x_0 + i y_0$.
		\begin{align*}
			u(x_0+h_1,y_0+h_2) = u(x_0,y_0) + u_xh_1 + u_y h_2 + o(h)
		\end{align*}

		and similarly for v. Then write $f(z_0 + h) - f(z_0)$ in terms of above and massage using cauchy riemann to get form $ah + o(h)$. 
	\end{proof}
\end{ans}

\begin{remark}
	Determinant of jacobian is really magnitude of norm of complex derivative squared.
\end{remark}

\begin{example}
	$f(x,y) = \sqrt{|x||y|}$ satisfies cauchy riemann but is  not holomorphic. 
\end{example}

\begin{theorem}
	Radius of convergence R of power series is
	\begin{align*}
		R = \frac{1}{limsup |a_n|^{1/n}}
	\end{align*}
\end{theorem}

\begin{proof}
	Idea is to compare to geometric series. Set R as desired. Then just compute(since we used limsup) and see that geometric series converges and or diverges in desired cases. This is also why we have problems on boundary. 
\end{proof}

\subsection{Power Series}

\textbf{TAG:} ComplexAnalysis

\begin{theorem}
	If f power series than $f'$ exists as $\sum n a_n z^{n-1}$ with same R, since $n^{1/n} \to 1$. 
\end{theorem}

\begin{proof}
	Fix $|z_0| < r < R$. Set $g(z) = \sum na_nz^{n-1}$. Write
	\begin{align*}
		\frac{f(z_0+h)-f(z_0)}{h} - g(z) = \frac{S_N(z_0+h)-S_N(z_0)}{h} - S_N'(z_0) + S_N'(z_0)-g(z_0) + \frac{E_N(z_0+h) - E_N(z_0)}{h}
	\end{align*}

	The first term is small using triangle inequality and $(z_0 + h)^n - z_0^h = h((z_0+h)^{n-1}+(z_0+h)^{n-2}z_0 + ...)$ which is bounded in norm by something still in R once h gets small enough. Thus this third term converges.

	The second term is small since this is a convergent power series.

	The first term is small by definition. And we are done.

	Notice this proof simultaneously asserts existence and shows what it is.
\end{proof}

\begin{proof}
	I had another proof by writing $f'(z_0 + h) = f(z_0) + ah + o(h)$ where $a = f'(z_0)$.
\end{proof}

\begin{remark}
	Power series are infinitely differentiable, since we have same disk of convergence. 
\end{remark}

\begin{quest}
	Is this how we establish holomorphic functions are infinitely differentiable? Cause they're all power series? How is this done?
\end{quest}

\begin{ans}
	Write holomorphic function as its taylor expansion and show they are close?
\end{ans}

\begin{theme}
	Anytime we prove something using algebraic facts, we have a complex argument since we haven't used the changed, rigid geometry at all.
\end{theme}

\begin{quest}
	What function is differentiable only once? Recall weierstrass cts everytwhere diff. nowhere
\end{quest}

\textbf{TAG:} ComplexIntegration

\begin{quest}
	Can we allow for a countable number of cuts? Will cutting in diff. ways countably lead to diff. integrals? Does this mess up notion of equivalency? What breaks? 
\end{quest}

\subsection{Complex Integration}

\begin{theorem}
	If $f$ has primitive then $\int_{\gamma} f(z)dz = F(\omega_2) - F(\omega_1)$
\end{theorem}

\begin{proof}
	Ports from real results by construction.
\end{proof}

\begin{example}
	$\int_{\gamma} dz/z=2\pi i$ where $\gamma$ is unit circle and hence $1/z$ must not have primitive(otherwise would be 0 on closed curve).
\end{example}

\begin{theorem}
	Gauss Lucas
\end{theorem}

\begin{proof}
	Exercise. Very algebraic. Probably write roots as convex combinations. Just compute by taking derivative

	In fact bring out missing roots by taking quotient $P'/P$. Easier to work with these than the complement set of sums and products(often working with quotients easier to access new roots).
\end{proof}

\begin{theorem}
	When $P(z) = \sum a_k a^k$ with real coeffiecients and only real zeroes then coefficients binomial(ultra) log concave.
\end{theorem}

\begin{proof}
	One line proof via gauss-lucas. So derivatives have real roots. Reciprocal poly also has real roots. Then to get the inequality consider $z^{n-k+1}P^{(k-1)}(1/z)$ which has real roots and take $n-k-1$ derivatives resulting in two deg poly. The coefficients of this thing gives desired result by looking at the discriminant.
\end{proof}

\begin{theme}
	This is common technique to show common sequences log-concave
\end{theme}

\begin{example}
	\begin{itemize}
		\item Stirling numbers of first kind
		\item Stirling numbers of second kind
		\item $t_k$ ultra log-concave where $t_k$ number of matchings of size $k$ in arbitrary graph G
	\end{itemize}
\end{example}

\begin{remark}
	To show mulitiplicative/additive property of exponential show $e^ze^(c-z)$ constant.
\end{remark}

\begin{theorem}
	(Goursout's Theorem): If $\Omega \subseteq \C$ open and $f : \Omega \to \C$ holomorphic with $\Delta \subseteq \Omega$ triangle then 
	\begin{align*}
		\int_{\partial \Delta} f = 0
	\end{align*}
\end{theorem}

\begin{proof}
	Bisect $\Delta$ noting
	\begin{align*}
		\int_{\partial \Delta^{(0)}} f = \int_{\partial \Delta^{(1)}_1} f + ... + \int_{\partial \Delta^{(1)}_4} f
	\end{align*}

	We continue recursively bisecting so that
	\begin{align*}
		|\int_{\partial \Delta^{(k)}} f| \leq 4 |\int_{\partal \Delta^{k+1}} f|
	\end{align*}

	The diameters $d_n \to 0$ and perimeters go to 0 so $\bigcap_{n=1}^{\infty} \Delta^{(n)} = \{z_0\}$(nested compact sets with diameter going to 0).

	We argue
	\begin{align*}
		4^n |\int_{\partial \Delta^{(n)}} f| \to 0
	\end{align*}

	since f is holomorphic. Write $f(z) = f(z_0) + f'(z_0)(z-z_0) + o(z)(z-z_0)$. Clearly $f(z_0)$ and $f'(z_0)(z-z_0)$ have primitives. Thus we are only actually integrating $(z-z_0)o(z)$. Upper bounding $(z-z_0)$ by diameter and $o(z)$ by some $sup_{\Delta^{(n)}} |f(z)|$ we have the bound $p_n d_n \sup$. Recall $p_n = 1/2p_{n-1}$ and $d_n = 1/2 d_{n-1}$ and so we're done.
\end{proof}

\begin{prop}
	Now since we have triangles we can also show for rectangles and by approximation most sets(tesselation). 
\end{prop}

\begin{theorem}
	Let $D \susbeteq \C$ be a disc, $f: D \to \C$ holomorphic then f has a primitive. 
\end{theorem}

\begin{proof}
	Suppose $D$ centered at 0. Write $F(z) = \int_{\gamma_z} f(w)dw$ where $\gamma_z$ is a triangular curve connecting z to 0. 

	Now compute with aim $F(z+h) - F(z) = hf(z)+ho(h) $
	\begin{align*}
		\int_{\gamma_{z+h}} f - \int_{\gamma_z}f = \int_{\gamma{z \to z+h}} f 
	\end{align*}

	by looking at paths and conidering goursout

	Then we conclude by using continuity 
\end{proof}

\begin{remark}
	Notice this argument relies on convexity. The problem is with holes(like the punctured disk).
\end{remark}

\end{document}

